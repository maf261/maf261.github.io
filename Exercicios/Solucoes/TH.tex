\documentclass{article}

\usepackage[brazil]{babel}
\usepackage[utf8]{inputenc}
%\UseRawInputEncoding
\usepackage[T1]{fontenc}
\usepackage{Sweave}
\usepackage{animate}
\usepackage{amsbsy}
\usepackage{amsfonts}
\usepackage{amsmath}
\usepackage{amssymb}
\usepackage{amsthm}
\usepackage[toc,page,title,titletoc]{appendix}
\usepackage[fixlanguage]{babelbib}
%\usepackage[pdftex]{color}
\usepackage{dsfont}
\usepackage{esvect}
\usepackage[labelfont=bf]{caption}
\usepackage{float}
\usepackage[Glenn]{fncychap}%Sonny %Conny %Lenny %Glenn %Renje %Bjarne %Bjornstrup
%\usepackage{geometry, calc, color, setspace}%
%\geometry{a4paper, headsep=1.0cm, footskip=1cm, lmargin=3cm, rmargin=2cm, tmargin=3cm, bmargin=2cm}
\usepackage{graphicx}
\usepackage{indentfirst}%Para indentar os parágrafos automáticamente
\usepackage{lipsum}
\usepackage{longtable}
\usepackage{mathtools}
\usepackage{listings}%Inserir codigo do R no latex
\usepackage{multirow}
\usepackage{multicol}
\usepackage{ifthen}
\newboolean{firstanswerofthechapter}  
\usepackage{natbib}
\bibliographystyle{abbrvnat3}
\usepackage[figuresright]{rotating}
\usepackage{spalign}
%\usepackage{pgfpages}
\usepackage{pgfplots}
\usepackage{tikz}
\pgfplotsset{compat=1.15}
\usepackage{mathrsfs}
\usetikzlibrary{arrows}
\usepackage{tasks}
\usepackage{color, colortbl}
\usepackage{xcolor}
\colorlet{lightcyan}{cyan!40!white}
\usepackage{chngcntr}
\usepackage{stackengine}
\usepackage{ragged2e}%para justificar o texto dentro de algum ambiente
\definecolor{Gray}{gray}{0.9}
\definecolor{LightCyan}{rgb}{0.88,1,1}

\usepackage[all]{xy}
\usepackage{hyperref,bookmark}
\hypersetup{
  colorlinks=true,
  linkcolor=blue,
  citecolor=red,
  filecolor=blue,
  urlcolor=blue,
}
\newlength{\longestlabel}
\settowidth{\longestlabel}{\bfseries viii.}

\setcounter{secnumdepth}{0} \setlength{\topmargin}{0cm}
\setlength{\headsep}{-0.3cm} \setlength{\textwidth}{17.5cm}
\setlength{\textheight}{23cm} \setlength{\oddsidemargin}{-0.8cm}
\setlength{\evensidemargin}{0cm} \setlength{\footskip}{-1.5cm}

\usepackage[lastexercise,answerdelayed]{exercise}
\renewcommand{\ExerciseName}{Exercícios}
\renewcommand{\ExerciseHeader}{\noindent\def\stackalignment{l}% code from https://tex.stackexchange.com/a/195118/101651
    \stackunder[0pt]{\colorbox{cyan}{\textcolor{white}{\textbf{\large\ExerciseName}}}}{\textcolor{lightcyan}{\rule{\linewidth}{2pt}}}\medskip}
    
% Partial code taken from http://mirrors.ctan.org/macros/latex/contrib/exercise/exercise.dtx
\newenvironment{ManualExercise}
  {\begin{list}{}{\leftmargin \QuestionIndent
    \partopsep0pt \parsep\parskip \topsep\QuestionBefore
    \itemsep\QuestionBefore \labelwidth2em
    \labelsep.33em
    \usecounter{Question}}}
  {\end{list}}    

\begin{document}

\Sconcordance{concordance:TH.tex:TH.Rnw:%
1 141 1 1 19 1 2 5 1 1 2 6 0 1 1 5 0 1 1 5 0 1 1 1 4 3 0 12 1 1 2 6 0 1 %
2 6 0 1 2 6 0 1 2 7 0 1 2 16 1 1 19 1 2 5 1 1 2 6 0 1 1 5 0 1 1 5 0 1 1 %
1 4 3 0 12 1 1 2 6 0 1 2 6 0 1 2 6 0 1 2 7 0 1 2 16 1 1 25 1 2 5 1 1 2 %
1 0 9 1 1 4 3 0 12 1 1 2 6 0 1 2 6 0 1 2 6 0 1 2 7 0 1 2 15 1 1 37 1 2 %
5 1 1 2 6 0 1 1 5 0 1 1 5 0 1 1 5 0 1 1 5 0 1 1 5 0 1 1 5 0 1 1 5 0 1 1 %
5 0 1 1 5 0 1 1 1 4 3 0 1 4 3 0 19 1 1 2 6 0 1 2 6 0 1 2 6 0 1 2 7 0 1 %
2 15 1 1 24 1 2 5 1 1 2 6 0 1 1 5 0 1 1 5 0 1 1 5 0 1 1 5 0 1 1 5 0 1 1 %
5 0 1 1 5 0 1 1 1 4 3 0 12 1 1 2 6 0 1 2 6 0 1 2 6 0 1 2 7 0 1 2 15 1 1 %
23 1 2 6 1 1 2 6 0 1 1 5 0 1 1 5 0 1 1 5 0 1 1 5 0 1 1 5 0 1 1 5 0 1 1 %
1 4 3 0 12 1 1 2 6 0 1 2 6 0 1 2 6 0 1 2 7 0 1 2 2 1 1 4 8 0 1 1 5 0 1 %
1 5 0 1 1 5 0 1 1 5 0 1 1 5 0 1 1 5 0 1 1 5 0 2 1 1 2 6 0 1 2 7 0 1 2 2 %
1 1 4 8 0 1 1 5 0 1 1 5 0 1 1 5 0 1 1 5 0 1 1 5 0 1 1 5 0 1 1 5 0 1 1 5 %
0 2 1 1 2 6 0 1 2 7 0 1 2 8 1 1 4 8 0 1 1 5 0 1 1 5 0 1 1 5 0 1 1 5 0 1 %
1 5 0 1 1 5 0 1 1 5 0 1 1 5 0 1 1 5 0 2 1 1 2 6 0 1 2 6 0 1 3 7 0 1 1 5 %
0 1 1 5 0 1 1 5 0 2 1 5 0 1 1 5 0 1 1 5 0 1 1 5 0 1 1 5 0 2 1 1 2 6 0 1 %
2 7 0 1 2 8 1 1 4 8 0 1 1 5 0 1 1 5 0 1 1 5 0 1 1 5 0 1 1 5 0 1 1 5 0 1 %
1 5 0 1 1 5 0 1 1 5 0 1 1 5 0 1 1 5 0 1 1 5 0 1 1 14 0 2 1 1 2 6 0 1 2 %
6 0 1 3 7 0 1 1 5 0 1 2 6 0 1 1 5 0 1 1 5 0 1 1 5 0 1 1 5 0 1 1 14 0 2 %
1 1 2 6 0 1 2 7 0 1 2 17 1 1 3 8 0 1 2 7 0 1 3 8 0 1 1 5 0 1 1 5 0 1 1 %
5 0 1 1 5 0 1 1 5 0 2 1 5 0 1 1 5 0 1 1 5 0 1 1 14 0 2 1 1 2 6 0 1 2 7 %
0 1 2 9 1 1 2 6 0 1 1 5 0 1 4 7 0 1 1 5 0 1 1 5 0 1 1 5 0 1 1 5 0 1 1 5 %
0 1 1 5 0 1 1 5 0 1 1 5 0 1 1 14 0 2 1 1 2 6 0 1 2 7 0 1 2 3 1 1 4 8 0 %
1 1 5 0 1 1 5 0 1 1 5 0 1 1 5 0 1 1 5 0 1 1 5 0 1 1 5 0 1 1 5 0 1 1 5 0 %
2 1 1 2 6 0 1 2 6 0 1 5 4 0 1 1 5 0 1 1 5 0 1 1 5 0 1 1 5 0 2 1 1 2 6 0 %
1 2 10 0 1 3 3 1 1 4 8 0 1 1 5 0 1 1 5 0 1 1 5 0 1 1 5 0 1 1 5 0 1 1 5 %
0 1 1 5 0 1 1 5 0 2 1 1 2 6 0 1 2 7 0 1 2 6 1 1 4 8 0 1 1 5 0 1 1 5 0 1 %
1 5 0 1 1 5 0 1 1 5 0 1 1 5 0 1 1 5 0 1 1 5 0 1 1 5 0 2 1 1 2 6 0 1 2 7 %
0 1 2 7 1 1 4 8 0 1 1 5 0 1 1 5 0 1 1 5 0 1 1 5 0 1 1 5 0 1 1 5 0 1 1 5 %
0 1 1 5 0 1 1 5 0 1 1 5 0 1 1 5 0 1 1 5 0 1 1 14 0 2 1 1 2 6 0 1 2 6 0 %
1 3 7 0 1 1 5 0 1 1 5 0 1 1 5 0 1 1 5 0 1 1 5 0 1 1 14 0 2 1 1 2 6 0 1 %
2 7 0 1 2 11 1 1 4 8 0 1 1 5 0 1 1 5 0 1 1 5 0 1 1 5 0 1 1 5 0 1 1 5 0 %
1 1 5 0 1 1 5 0 1 1 5 0 1 1 5 0 1 1 5 0 1 1 5 0 1 1 14 0 2 1 1 2 6 0 1 %
2 6 0 1 3 7 0 1 1 5 0 1 1 5 0 1 1 5 0 1 1 5 0 1 1 5 0 1 1 5 0 1 1 5 0 1 %
1 14 0 2 1 1 2 6 0 1 2 7 0 1 2 4 1}


\vspace*{-2cm}

\begin{center}
\begin{minipage}[s]{2cm}
\hspace{-1.3cm}\includegraphics[scale=1.0]{Figuras/brasaoufv.eps}
\end{minipage}
\begin{minipage}[s]{13cm}
{\begin{center} {\sc \Large Universidade Federal de Vi\c{c}osa}\\
{\sc \large Instituto de Ci\^encias Exatas e Tecnológicas}\\
{\sc \large Campus UFV - Florestal}\\
\end{center}}
\end{minipage}\begin{minipage}[s]{2 cm}
%\includegraphics[width=2 cm]{logoimecc.eps}
\end{minipage}
\end{center}

\vspace{-0.3cm}

%\hline \hline \noindent

%%%%%%%%%%%%%%%%%%%%%%%%%%%%%%%%%%%%%%%%%%%%%%%%%%%%%%%%%%%%%%%%%%%%%%%%%%%

\medskip

\begin{center}

\underline{\underline{{\large{\sc Soluções da Lista de Testes de Hipóteses}}}}

\bigskip

{\large {\bf Prof. Fernando Bastos}}
%\bigskip
%
%%{\sc Data: $19/06/2018$}
\end{center}


\begin{Exercise}
\begin{ManualExercise} 
\item[9.~]O fabricante de certa marca de suco informa que as embalagens de seu produto têm em média 500 ml, com desvio padrão igual a 10 ml. Tendo sido encontradas no mercado algumas embalagens com menos de 500 ml, suspeita-se que a informação do fabricante seja falsa. Para verificar se isto ocorre, um fiscal analisa uma amostra de 200 embalagens escolhidas aleatoriamente no mercado e constata que as mesmas contêm em média 498 ml. Considerando-se um nível de significância de $5\%,$ pode-se afirmar que o fabricante está mentindo? Calcule o valor da prova para esta amostra.

\begin{align*}
H_{0}: \mu&=500ml \\ 
H_{1}: \mu&<500ml\quad \textrm{(unilateral)}
\end{align*}

Dados:

$n=200;\quad \bar{x}=498ml;\quad \sigma=10ml;\quad \alpha=5\% \rightarrow z_{t}=-1,64$

$$z_{cal}=\dfrac{\bar{x}-\mu_{0}}{\dfrac{\sigma}{\sqrt{n}}}=\dfrac{498-500}{10/\sqrt{200}}=-2.83$$
$$p-valor=0.002338867$$
\begin{center}
\setkeys{Gin}{width=0.5\linewidth}
\includegraphics{TH-001}
\end{center}

Decisão: Como $|z_{cal}|>|z_{tab}|$ rejeita-se $H_{0}$ ao nível $\alpha=5\%$ de significância.

Comandos em R para soluções:

\begin{Schunk}
\begin{Sinput}
> (zt <- qnorm(0.05))
\end{Sinput}
\begin{Soutput}
[1] -1.644854
\end{Soutput}
\begin{Sinput}
> (zc <- (498-500)/(10/sqrt(200)))
\end{Sinput}
\begin{Soutput}
[1] -2.828427
\end{Soutput}
\begin{Sinput}
> (pvalor <- pnorm(zc))
\end{Sinput}
\begin{Soutput}
[1] 0.002338867
\end{Soutput}
\begin{Sinput}
> curve(dnorm(x), from=-4.5, to=4.5, xlab="", ylab="")
> polygon(cbind(c(zt,seq(zt,-4.5, l=100),-4.5), 
+               c(0, dnorm(seq(zt, -4.5, l=100)), 
+                 dnorm(-4.5))), 
+         col="lightgray")
> abline(v=zt, lty=2)
> arrows(zc, 0.1, zc, 0)
> zt <- format(zt,digits = 3)
> Zt <- bquote(bold(z[t] == .(zt)))
> zc <- format(zc,digits = 3)
> Zc <- bquote(bold(z[c] == .(zc)))
> text(zt, 0.1, Zt, pos=4)
> text(zt, 0.2, "RNRH0", pos=4)
> text(zc, 0.12, Zc, pos=3)
> text(zc, 0.2, "RRH0", pos=3)
> RR <- "Rejeita-se H0 ao nível de 5% de significância"
> RN <- "Não rejeita-se H0 ao nível de 5% de significância"
> ##Resultado
> ifelse(zc>zt,RR,RN)
\end{Sinput}
\begin{Soutput}
[1] "Rejeita-se H0 ao nível de 5% de significância"
\end{Soutput}
\begin{Sinput}
> ##Ou, equivalentemente:
> ifelse(pvalor > 0.05, RN, RR)
\end{Sinput}
\begin{Soutput}
[1] "Rejeita-se H0 ao nível de 5% de significância"
\end{Soutput}
\begin{Sinput}
> ## estimativa pontual
> (mu.est <- 498)
\end{Sinput}
\begin{Soutput}
[1] 498
\end{Soutput}
\begin{Sinput}
> ## estimativa intervalar (95%)
> (IC.mu <- mu.est + qnorm(c(0.025, 0.975)) * 10/sqrt(200))
\end{Sinput}
\begin{Soutput}
[1] 496.6141 499.3859
\end{Soutput}
\end{Schunk}


\item[10.~]A duração das lâmpadas produzidas por certo fabricante tem distribuição normal com média igual a 1200 horas e desvio padrão igual a 300 horas. O fabricante introduz um novo processo na produção das lâmpadas. Para verificar se o novo processo produz lâmpadas de maior duração, o fabricante observa 100 lâmpadas produzidas pelo novo processo e constata que as mesmas duram em média 1265 horas. Admitindo-se um nível de significância de $5\%,$ pode-se concluir que o novo processo produz lâmpadas com maior duração?

\begin{align*}
H_{0}: \mu&=1200h \\ 
H_{1}: \mu&>1200h\quad \textrm{(unilateral)}
\end{align*}

Dados:

$n=100;\quad \bar{x}=1265h;\quad \sigma=300h;\quad \alpha=5\% \rightarrow z_{t}=1,64$

$$z_{cal}=\dfrac{\bar{x}-\mu_{0}}{\dfrac{\sigma}{\sqrt{n}}}=\dfrac{1265-1200}{300/\sqrt{100}}=2,17$$
$$p-valor=0.01513014$$
\begin{center}
\setkeys{Gin}{width=0.5\linewidth}
\includegraphics{TH-003}
\end{center}

Decisão: Como $|z_{cal}|>|z_{tab}|$ rejeita-se $H_{0}$ ao nível $\alpha=5\%$ de significância.

Comandos em R para soluções:

\begin{Schunk}
\begin{Sinput}
> (zt <- qnorm(0.95))
\end{Sinput}
\begin{Soutput}
[1] 1.644854
\end{Soutput}
\begin{Sinput}
> (zc <- (1265-1200)/(300/sqrt(100)))
\end{Sinput}
\begin{Soutput}
[1] 2.166667
\end{Soutput}
\begin{Sinput}
> (pvalor <- 1-pnorm(zc))
\end{Sinput}
\begin{Soutput}
[1] 0.01513014
\end{Soutput}
\begin{Sinput}
> curve(dnorm(x), from=-4.5, to=4.5, xlab="", ylab="")
> polygon(cbind(c(zt,seq(zt,4.5, l=100),4.5), 
+               c(0, dnorm(seq(zt, 4.5, l=100)), 
+                 dnorm(4.5))), 
+         col="lightgray")
> abline(v=zt, lty=2)
> arrows(zc, 0.1, zc, 0)
> zt <- format(zt,digits = 3)
> Zt <- bquote(bold(z[t] == .(zt)))
> zc <- format(zc,digits = 3)
> Zc <- bquote(bold(z[c] == .(zc)))
> text(zt, 0.1, Zt, pos=2)
> text(zt, 0.2, "RNRH0", pos=2)
> text(zc, 0.12, Zc, pos=4)
> text(zc, 0.2, "RRH0", pos=4)
> RR <- "Rejeita-se H0 ao nível de 5% de significância"
> RN <- "Não rejeita-se H0 ao nível de 5% de significância"
> ##Resultado
> ifelse(zc>zt,RR,RN)
\end{Sinput}
\begin{Soutput}
[1] "Rejeita-se H0 ao nível de 5% de significância"
\end{Soutput}
\begin{Sinput}
> ##Ou, equivalentemente:
> ifelse(pvalor > 0.05, RN, RR)
\end{Sinput}
\begin{Soutput}
[1] "Rejeita-se H0 ao nível de 5% de significância"
\end{Soutput}
\begin{Sinput}
> ## estimativa pontual
> (mu.est <- 1265)
\end{Sinput}
\begin{Soutput}
[1] 1265
\end{Soutput}
\begin{Sinput}
> ## estimativa intervalar (95%)
> (IC.mu <- mu.est + qnorm(c(0.025, 0.975)) * 300/sqrt(100))
\end{Sinput}
\begin{Soutput}
[1] 1206.201 1323.799
\end{Soutput}
\end{Schunk}

\item[11.~]O custo de produção de certo artigo numa localidade tem distribuição normal com média igual a $R\$42,00.$ Desenvolve-se uma política de redução de custos na empresa para melhorar a competitividade do referido produto no mercado. Observando-se os custos de 10 unidades deste produto, obtiveram-se os seguintes valores: 
$34, 41, 36, 41, 29, 32, 38, 35, 33 e 30.$ Admitindo-se um nível de significância de $5\%,$ pode-se afirmar que o custo do produto considerado diminuiu? 

\begin{align*}
H_{0}: \mu&=R\$42,00 \\
H_{1}: \mu&<R\$42,00\quad \textrm{(unilateral)}
\end{align*}

Dados:

$n=10;\quad \bar{x}=34,9;\quad S=4,17;\quad \alpha=5\% \rightarrow t_{t}=-1.83$

$$t_{cal}=\dfrac{\bar{x}-\mu_{0}}{\dfrac{S}{\sqrt{n}}}=\dfrac{34,9-42}{4,17/\sqrt{10}}=-5,38$$
$$p-valor=0.0002210237$$
\begin{center}
\setkeys{Gin}{width=0.5\linewidth}
\includegraphics{TH-005}
\end{center}

Decisão: Como $|t_{cal}|>|t_{tab}|$ rejeita-se $H_{0}$ ao nível $\alpha=5\%$ de significância.
%Como $|z_{cal}|>|z_{tab}|$ não rejeita-se $H_{0}$ ao nível $\alpha=AQUI\%$ de significância.
Comandos em R para soluções:

\begin{Schunk}
\begin{Sinput}
> x <- c(34, 41, 36, 41, 29, 32, 38, 35, 33, 30)
> n=length(x)
> df=n-1
> alpha=0.05
> barx <- mean(x)
> dp <- sd(x)
> tt <- qt(alpha,df)
> tc <- (34.9-42)/(4.17/sqrt(10))
> pvalor <- pt(tc,df)
> curve(dt(x,df), from=-6, to=6, xlab="", ylab="")
> polygon(cbind(c(tt,seq(tt,-6, l=100),-6),
+               c(0, dt(seq(tt, -6, l=100),df),
+                 dt(-6,df))),
+         col="lightgray")
> abline(v=tt, lty=2)
> arrows(tc, 0.1, tc, 0)
> tt <- format(tt,digits = 3)
> Tt <- bquote(bold(t[t] == .(tt)))
> tc <- format(tc,digits = 3)
> Tc <- bquote(bold(t[c] == .(tc)))
> text(tt, 0.1, Tt, pos=4)
> text(tt, 0.2, "RNRH0", pos=4)
> text(tc, 0.12, Tc, pos=3)
> text(tc, 0.2, "RRH0", pos=4)
> RR <- "Rejeita-se H0 ao nível de 5% de significância"
> RN <- "Não rejeita-se H0 ao nível de 5% de significância"
> ##Resultado
> ifelse(tc>tt,RR,RN)
\end{Sinput}
\begin{Soutput}
[1] "Rejeita-se H0 ao nível de 5% de significância"
\end{Soutput}
\begin{Sinput}
> ##Ou, equivalentemente:
> ifelse(pvalor > 0.05, RN, RR)
\end{Sinput}
\begin{Soutput}
[1] "Rejeita-se H0 ao nível de 5% de significância"
\end{Soutput}
\begin{Sinput}
> ## estimativa pontual
> (mu.est <- 34.9)
\end{Sinput}
\begin{Soutput}
[1] 34.9
\end{Soutput}
\begin{Sinput}
> ## estimativa intervalar (95%)
> (IC.mu <- mu.est + qt(c(0.025, 0.975),df) * 4.17/sqrt(10))
\end{Sinput}
\begin{Soutput}
[1] 31.91696 37.88304
\end{Soutput}
\end{Schunk}

\item[12.~]O controle de qualidade das peças produzidas por certa fábrica exige que o diâmetro médio das mesmas seja $57$ mm. Para verificar se o processo de produção está sob controle, observam-se os diâmetros de $10$ peças, constatando-se os seguintes valores em mm: $56,5; 56,6; 57,3; 56,9; 57,1; 56,7; 57,1; 56,8; 57,1; 57,0.$ Admitindo-se um nível de significância de $5\%,$ pode-se concluir que o processo de produção está sob controle?

\begin{align*}
H_{0}: \mu&=57mm \\
H_{1}: \mu&\neq 57mm\quad \textrm{(bilateral)}
\end{align*}

Dados:

$n=10;\quad \bar{x}=56.91;\quad S=0.256;\quad \alpha=0.05\% \rightarrow t_{t}=2.262157$

$$t_{cal}=\dfrac{\bar{x}-\mu_{0}}{\dfrac{S}{\sqrt{n}}}=\dfrac{56.91-57}{0.256/\sqrt{10}}=-1.112$$
$$p-valor=0.2947482$$
\begin{center}
\setkeys{Gin}{width=0.5\linewidth}
\includegraphics{TH-007}
\end{center}

Decisão: %Como $|t_{cal}|>|t_{tab}|$ rejeita-se $H_{0}$ ao nível $\alpha=AQUI\%$ de significância.
%Como $|z_{cal}|>|z_{tab}|$ não rejeita-se $H_{0}$ ao nível $\alpha=AQUI\%$ de significância.
Comandos em R para soluções:

\begin{Schunk}
\begin{Sinput}
> (x <- c(56.5, 56.6, 57.3, 56.9, 57.1, 56.7, 57.1, 56.8, 57.1, 57.0))
\end{Sinput}
\begin{Soutput}
 [1] 56.5 56.6 57.3 56.9 57.1 56.7 57.1 56.8 57.1 57.0
\end{Soutput}
\begin{Sinput}
> (n=length(x))
\end{Sinput}
\begin{Soutput}
[1] 10
\end{Soutput}
\begin{Sinput}
> (df=n-1)
\end{Sinput}
\begin{Soutput}
[1] 9
\end{Soutput}
\begin{Sinput}
> (alpha=0.025)
\end{Sinput}
\begin{Soutput}
[1] 0.025
\end{Soutput}
\begin{Sinput}
> (barx <- mean(x))
\end{Sinput}
\begin{Soutput}
[1] 56.91
\end{Soutput}
\begin{Sinput}
> (dp <- sd(x))
\end{Sinput}
\begin{Soutput}
[1] 0.2558211
\end{Soutput}
\begin{Sinput}
> (tt <- qt(alpha,df))
\end{Sinput}
\begin{Soutput}
[1] -2.262157
\end{Soutput}
\begin{Sinput}
> (mu <- 57)
\end{Sinput}
\begin{Soutput}
[1] 57
\end{Soutput}
\begin{Sinput}
> (tc <- (barx-mu)/(dp/sqrt(n)))
\end{Sinput}
\begin{Soutput}
[1] -1.112516
\end{Soutput}
\begin{Sinput}
> (pvalor <- 2*pt(tc,df))
\end{Sinput}
\begin{Soutput}
[1] 0.2947482
\end{Soutput}
\begin{Sinput}
> curve(dt(x,df), from=-6, to=6, xlab="", ylab="")
> polygon(cbind(c(-abs(tt),seq(-abs(tt),-6, l=100),-6),
+               c(0, dt(seq(-abs(tt), -6, l=100),df),
+                 dt(-6,df))),
+         col="lightgray")
> polygon(cbind(c(abs(tt),seq(abs(tt),6, l=100),6),
+               c(dt(6,df), dt(seq(abs(tt), 6, l=100),df),
+                 0)),
+         col="lightgray")
> abline(v=tt, lty=2)
> abline(v=-tt, lty=2)
> arrows(tc, 0.1, tc, 0)
> tt1 <- qt(alpha,df)
> tt1 <- format(tt1,digits = 3)
> Tt1 <- bquote(bold(t[t] == .(tt1)))
> tt2 <- -qt(alpha,df)
> tt2 <- format(tt2,digits = 3)
> Tt2 <- bquote(bold(t[t] == .(tt2)))
> tc1 <- format(tc,digits = 3)
> Tc1 <- bquote(bold(t[c] == .(tc1)))
> text(tt1, 0.1, Tt1, pos=2)
> text(tt1, 0.2, "RRH0", pos=2)
> text(tt2, 0.1, Tt2, pos=4)
> text(tt2, 0.2, "RRH0", pos=4)
> text(tc1, 0.1, Tc1, pos=4)
> text(tc1, 0.2, "RNRH0", pos=4)
> RR <- "Rejeita-se H0 ao nível de 5% de significância"
> RN <- "Não rejeita-se H0 ao nível de 5% de significância"
> ##Resultado
> (ifelse((tc<tt || tc>(abs(tt))),RR,RN))
\end{Sinput}
\begin{Soutput}
[1] "Não rejeita-se H0 ao nível de 5% de significância"
\end{Soutput}
\begin{Sinput}
> ##Ou, equivalentemente:
> (ifelse(pvalor > ns, RN, RR))
\end{Sinput}
\begin{Soutput}
[1] "Não rejeita-se H0 ao nível de 5% de significância"
\end{Soutput}
\begin{Sinput}
> ## estimativa pontual
> (mu.est <- barx)
\end{Sinput}
\begin{Soutput}
[1] 56.91
\end{Soutput}
\begin{Sinput}
> ## estimativa intervalar (95%)
> (IC.mu <- mu.est + qt(c((alpha/2), (1-alpha/2)),df) * (dp/sqrt(n)))
\end{Sinput}
\begin{Soutput}
[1] 56.69279 57.12721
\end{Soutput}
\end{Schunk}

\item[13.~]Numa localidade, $32\%$ dos consumidores consomem determinado produto. Foi lançado no mercado da localidade um produto concorrente. Uma pesquisa realizada com 500 consumidores escolhidos ao acaso revelou que 145 dentre estes consomem o antigo produto. Pode-se concluir, num nível de significância de $2\%,$ que a preferência pelo produto antigo diminuiu com a entrada do concorrente no mercado? Calcule o valor da prova para esta amostra.

\begin{align*}
H_{0}: p&=0.32 \\
H_{1}: p&\neq 0.32\quad \textrm{(bilateral)}
\end{align*}

Dados:

$n=500;\quad \bar{x}=145;\quad p_{0}=0.32;\quad \alpha=2\% \rightarrow z_{t}=-2.326$

$$z_{cal}=\dfrac{\bar{x}-np_{0}}{\sqrt{np_{0}(1-p_{0})}}=\dfrac{145-160}{\sqrt{160(0.68)}}=-1.43$$
$$p-valor=0.1504172$$
\begin{center}
\setkeys{Gin}{width=0.5\linewidth}
\includegraphics{TH-009}
\end{center}

Decisão: Como %$|z_{cal}|>|z_{tab}|$ rejeita-se $H_{0}$ ao nível $\alpha=AQUI\%$ de significância.
$|z_{cal}|<|z_{tab}|$ não rejeita-se $H_{0}$ ao nível $\alpha=2\%$ de significância.
Comandos em R para soluções:

\begin{Schunk}
\begin{Sinput}
> (ns <- 0.02)
\end{Sinput}
\begin{Soutput}
[1] 0.02
\end{Soutput}
\begin{Sinput}
> (alpha <- 0.01)
\end{Sinput}
\begin{Soutput}
[1] 0.01
\end{Soutput}
\begin{Sinput}
> (n <- 500)
\end{Sinput}
\begin{Soutput}
[1] 500
\end{Soutput}
\begin{Sinput}
> (p0 <- 0.32)
\end{Sinput}
\begin{Soutput}
[1] 0.32
\end{Soutput}
\begin{Sinput}
> (barx <- 145)
\end{Sinput}
\begin{Soutput}
[1] 145
\end{Soutput}
\begin{Sinput}
> (zt <- qnorm(alpha))
\end{Sinput}
\begin{Soutput}
[1] -2.326348
\end{Soutput}
\begin{Sinput}
> (zc <- (barx-(n*p0))/(sqrt(n*p0*(1-p0))))
\end{Sinput}
\begin{Soutput}
[1] -1.438059
\end{Soutput}
\begin{Sinput}
> (pvalor <- 2*pnorm(zc))
\end{Sinput}
\begin{Soutput}
[1] 0.1504172
\end{Soutput}
\begin{Sinput}
> curve(dnorm(x), from=-4.5, to=4.5, xlab="", ylab="")
> polygon(cbind(c(-4.5, seq(-4.5,zt, l=100),zt),
+               c(0, dnorm(seq(-4.5, zt, l=100)),
+                 (0))),
+         col="lightgray")
> polygon(cbind(c(abs(zt), seq(abs(zt),4.5, l=100),4.5),
+               c(0, dnorm(seq(abs(zt), 4.5, l=100)),
+                 0)),
+         col="lightgray")
> abline(v=zt, lty=2)
> abline(v=abs(zt), lty=2)
> arrows(zc, 0.1, zc, 0)
> zt1 <- format(zt,digits = 3)
> Zt1 <- bquote(bold(z[t] == .(zt1)))
> zt2 <- format(abs(zt),digits = 3)
> Zt2 <- bquote(bold(z[t] == .(zt2)))
> zc1 <- format(zc,digits = 3)
> Zc1 <- bquote(bold(z[c] == .(zc1)))
> text(zt1, 0.1, Zt, pos=2)
> text(zt1, 0.2, "RRH0", pos=2)
> text(zt2, 0.1, Zt2, pos=4)
> text(zt2, 0.2, "RRH0", pos=4)
> text(zc1, 0.12, Zc1, pos=4)
> text(zc1, 0.2, "RNRH0", pos=4)
> RR <- "Rejeita-se H0 ao nível de 2% de significância"
> RN <- "Não rejeita-se H0 ao nível de 2% de significância"
> ##Resultado
> ifelse((zc<zt || zc>abs(zt)),RR,RN)
\end{Sinput}
\begin{Soutput}
[1] "Não rejeita-se H0 ao nível de 2% de significância"
\end{Soutput}
\begin{Sinput}
> ##Ou, equivalentemente:
> ifelse(pvalor > 0.02, RN, RR)
\end{Sinput}
\begin{Soutput}
[1] "Não rejeita-se H0 ao nível de 2% de significância"
\end{Soutput}
\begin{Sinput}
> ## estimativa pontual
> (mu.est <- 145)
\end{Sinput}
\begin{Soutput}
[1] 145
\end{Soutput}
\begin{Sinput}
> ## estimativa intervalar (98%)
> (IC.mu <- mu.est + qnorm(c(0.01, 0.99)) * (n*p0)/sqrt(n*p0*(1-p0)))
\end{Sinput}
\begin{Soutput}
[1] 109.3155 180.6845
\end{Soutput}
\end{Schunk}

\item[14.~]Sabe-se que $6\%$ das unidades de certo produto são substituídas gratuitamente por apresentar defeitos de fabricação. Para reduzir este percentual, o fabricante investiu na melhoria da qualidade do produto. Consta-se que 12 dentre 400 unidades vendidas tiveram que ser substituídas gratuitamente por apresentar defeitos de fabricação. Pode-se concluir, num nível de significância de $3\%,$ que a qualidade do produto melhorou?

\begin{align*}
H_{0}: p&=0.06 \\
H_{1}: p&< 0.06\quad \textrm{(unilateral)}
\end{align*}

Dados:

$n=400;\quad \bar{x}=12;\quad p_{0}=0.06;\quad \alpha=3\% \rightarrow z_{t}=-1.88$

$$z_{cal}=\dfrac{\bar{x}-np_{0}}{\sqrt{np_{0}(1-p_{0})}}=\dfrac{12-24}{\sqrt{24(0.94)}}=-2.526$$
$$p-valor=0.005760995$$
\begin{center}
\setkeys{Gin}{width=0.5\linewidth}
\includegraphics{TH-011}
\end{center}

Decisão: Como $|z_{cal}|>|z_{tab}|$ rejeita-se $H_{0}$ ao nível $\alpha=3\%$ de significância.
%$|z_{cal}|<|z_{tab}|$ não rejeita-se $H_{0}$ ao nível $\alpha=2\%$ de significância.

Comandos em R para soluções:

\begin{Schunk}
\begin{Sinput}
> (alpha <- 0.03)
\end{Sinput}
\begin{Soutput}
[1] 0.03
\end{Soutput}
\begin{Sinput}
> (n <- 400)
\end{Sinput}
\begin{Soutput}
[1] 400
\end{Soutput}
\begin{Sinput}
> (p0 <- 0.06)
\end{Sinput}
\begin{Soutput}
[1] 0.06
\end{Soutput}
\begin{Sinput}
> (barx <- 12)
\end{Sinput}
\begin{Soutput}
[1] 12
\end{Soutput}
\begin{Sinput}
> (zt <- qnorm(alpha))
\end{Sinput}
\begin{Soutput}
[1] -1.880794
\end{Soutput}
\begin{Sinput}
> (zc <- (barx-(n*p0))/(sqrt(n*p0*(1-p0))))
\end{Sinput}
\begin{Soutput}
[1] -2.526456
\end{Soutput}
\begin{Sinput}
> (pvalor <- pnorm(zc))
\end{Sinput}
\begin{Soutput}
[1] 0.005760995
\end{Soutput}
\begin{Sinput}
> curve(dnorm(x), from=-4.5, to=4.5, xlab="", ylab="")
> polygon(cbind(c(-4.5, seq(-4.5,zt, l=100),zt),
+               c(0, dnorm(seq(-4.5, zt, l=100)),
+                 (0))),
+         col="lightgray")
> abline(v=zt, lty=2)
> arrows(zc, 0.1, zc, 0)
> zt1 <- format(zt,digits = 3)
> Zt1 <- bquote(bold(z[t] == .(zt1)))
> zc1 <- format(zc,digits = 3)
> Zc1 <- bquote(bold(z[c] == .(zc1)))
> text(zt1, 0.1, Zt, pos=4)
> text(zt1, 0.2, "RNRH0", pos=4)
> text(zc1, 0.12, Zc1, pos=2)
> text(zc1, 0.2, "RRH0", pos=2)
> RR <- "Rejeita-se H0 ao nível de 3% de significância"
> RN <- "Não rejeita-se H0 ao nível de 3% de significância"
> ##Resultado
> ifelse((zc<zt || zc>abs(zt)),RR,RN)
\end{Sinput}
\begin{Soutput}
[1] "Rejeita-se H0 ao nível de 3% de significância"
\end{Soutput}
\begin{Sinput}
> ##Ou, equivalentemente:
> ifelse(pvalor > 0.03, RN, RR)
\end{Sinput}
\begin{Soutput}
[1] "Rejeita-se H0 ao nível de 3% de significância"
\end{Soutput}
\begin{Sinput}
> ## estimativa pontual
> (mu.est <- 12)
\end{Sinput}
\begin{Soutput}
[1] 12
\end{Soutput}
\begin{Sinput}
> ## estimativa intervalar (98%)
> (IC.mu <- mu.est + qnorm(c(0.015, 0.985)) * (n*p0)/sqrt(n*p0*(1-p0)))
\end{Sinput}
\begin{Soutput}
[1]  1.034725 22.965275
\end{Soutput}
\end{Schunk}

\item[18.~]Uma fábrica de automóveis anuncia que seus carros consomem, em média, 11 litros por 100 km, com desvio padrão de 0,8 litros. Uma revista resolve testar essa afirmação e analisa 35 automóveis dessa marca, obtendo 11,3 litros por 100 km como consumo médio (considerar distribução normal). O que a revista pode concluir sobre o anúncio da fábrica, no nível de $10\%?$

\begin{Schunk}
\begin{Sinput}
> #H0:\mu=0.11km/l
> #H1:\mu!=0.11km/l
> (mu <- 0.11)
\end{Sinput}
\begin{Soutput}
[1] 0.11
\end{Soutput}
\begin{Sinput}
> (sigma <- 0.8)
\end{Sinput}
\begin{Soutput}
[1] 0.8
\end{Soutput}
\begin{Sinput}
> (n <- 35)
\end{Sinput}
\begin{Soutput}
[1] 35
\end{Soutput}
\begin{Sinput}
> (barx <- 0.113)
\end{Sinput}
\begin{Soutput}
[1] 0.113
\end{Soutput}
\begin{Sinput}
> (alpha <- 0.1)
\end{Sinput}
\begin{Soutput}
[1] 0.1
\end{Soutput}
\begin{Sinput}
> (zc <- (barx-mu)/(sigma/sqrt(n)))
\end{Sinput}
\begin{Soutput}
[1] 0.0221853
\end{Soutput}
\begin{Sinput}
> (zt <- qnorm(0.05)) #Teste bilateral
\end{Sinput}
\begin{Soutput}
[1] -1.644854
\end{Soutput}
\begin{Sinput}
> (pvalor <- 2*pnorm(zc))
\end{Sinput}
\begin{Soutput}
[1] 1.0177
\end{Soutput}
\begin{Sinput}
> RR <- "Rejeita-se H0 ao nível de 10% de significância"
> RN <- "Não rejeita-se H0 ao nível de 10% de significância"
> ##Resultado
> ifelse(abs(zc)>abs(zt),RR,RN)
\end{Sinput}
\begin{Soutput}
[1] "Não rejeita-se H0 ao nível de 10% de significância"
\end{Soutput}
\begin{Sinput}
> ##Ou, equivalentemente:
> ifelse(pvalor > 0.1, RN, RR)
\end{Sinput}
\begin{Soutput}
[1] "Não rejeita-se H0 ao nível de 10% de significância"
\end{Soutput}
\end{Schunk}

\item[19.~]Duas máquinas, A e B, são usadas para empacotar pó de café. A experiência passada garante que o desvio padrão para ambas é de 10g. Porém, suspeita-se que elas têm médias diferentes. Para verificar, sortearam-se duas amostras: uma com 25 pacotes da máquina A e outra com 16 pacotes da máquina B. As médias foram, respectivamente, $\bar{x}_{A} = 502,74g$ e $\bar{x}_{B} = 496,60 g$.  Com esses números, e com o nível de $5\%$, qual seria a coclusão do teste $H_0:\mu_A =\mu_B$?

\begin{Schunk}
\begin{Sinput}
> #H0:\muA=\muB
> #H1:\muA!=\muB
> (barxA <- 502.74)
\end{Sinput}
\begin{Soutput}
[1] 502.74
\end{Soutput}
\begin{Sinput}
> (barxB <- 496.60)
\end{Sinput}
\begin{Soutput}
[1] 496.6
\end{Soutput}
\begin{Sinput}
> (sigma <- 10)
\end{Sinput}
\begin{Soutput}
[1] 10
\end{Soutput}
\begin{Sinput}
> (nA <- 25)
\end{Sinput}
\begin{Soutput}
[1] 25
\end{Soutput}
\begin{Sinput}
> (nB <- 16)
\end{Sinput}
\begin{Soutput}
[1] 16
\end{Soutput}
\begin{Sinput}
> (alpha <- 0.05)
\end{Sinput}
\begin{Soutput}
[1] 0.05
\end{Soutput}
\begin{Sinput}
> (zc <- (barxA-barxB)/(sigma*sqrt((1/nA)+(1/nB))))
\end{Sinput}
\begin{Soutput}
[1] 1.917814
\end{Soutput}
\begin{Sinput}
> (zt <- qnorm(0.025)) #Teste bilateral
\end{Sinput}
\begin{Soutput}
[1] -1.959964
\end{Soutput}
\begin{Sinput}
> (pvalor <- 2*pnorm(zc))
\end{Sinput}
\begin{Soutput}
[1] 1.944865
\end{Soutput}
\begin{Sinput}
> RR <- "Rejeita-se H0 ao nível de 5% de significância"
> RN <- "Não rejeita-se H0 ao nível de 5% de significância"
> ##Resultado
> ifelse(abs(zc)>abs(zt),RR,RN)
\end{Sinput}
\begin{Soutput}
[1] "Não rejeita-se H0 ao nível de 5% de significância"
\end{Soutput}
\begin{Sinput}
> ##Ou, equivalentemente:
> ifelse(pvalor > 0.05, RN, RR)
\end{Sinput}
\begin{Soutput}
[1] "Não rejeita-se H0 ao nível de 5% de significância"
\end{Soutput}
\end{Schunk}

\item[20.~]Uma fábrica de embalagens para produtos químicos está estudando dois processos para combater a corrosão de suas latas especiais. Para verificar o efeito dos tratamentos, foram usadas amostras cujos resultados estão no quadro abaixo (em porcentagem de corrosão eliminada). Qual seria a conclusão sobre os dois tratamentos?

\begin{tabular}{cccc}\\ \hline
Método & Amostra & Média & Desvio Padrão \\ \hline
A & 15 & 48 & 10 \\
B & 12 & 52 & 15 \\ \hline
\end{tabular}

\begin{Schunk}
\begin{Sinput}
> #H0:\sigma_{A}^{2}=\sigma_{B}^{2}
> #H1:\sigma_{A}^{2}<\sigma_{B}^{2}
> (dpA <- 10)
\end{Sinput}
\begin{Soutput}
[1] 10
\end{Soutput}
\begin{Sinput}
> (dpB <- 15)
\end{Sinput}
\begin{Soutput}
[1] 15
\end{Soutput}
\begin{Sinput}
> (nA <- 15)
\end{Sinput}
\begin{Soutput}
[1] 15
\end{Soutput}
\begin{Sinput}
> (dfA <- nA-1)
\end{Sinput}
\begin{Soutput}
[1] 14
\end{Soutput}
\begin{Sinput}
> (nB <- 12)
\end{Sinput}
\begin{Soutput}
[1] 12
\end{Soutput}
\begin{Sinput}
> (dfB <- nB-1)
\end{Sinput}
\begin{Soutput}
[1] 11
\end{Soutput}
\begin{Sinput}
> (alpha <- 0.05)
\end{Sinput}
\begin{Soutput}
[1] 0.05
\end{Soutput}
\begin{Sinput}
> (fc <- (dpA^2)/(dpB^2))
\end{Sinput}
\begin{Soutput}
[1] 0.4444444
\end{Soutput}
\begin{Sinput}
> (ft <- qf(alpha,dfA,dfB))
\end{Sinput}
\begin{Soutput}
[1] 0.389788
\end{Soutput}
\begin{Sinput}
> (pvalor <-(pf(fc,dfA,dfB)))
\end{Sinput}
\begin{Soutput}
[1] 0.07754768
\end{Soutput}
\begin{Sinput}
> RR <- "Rejeita-se H0 ao nível alpha=5% de significância"
> RN <- "Não rejeita-se H0 ao nível alpha=5% de significância"
> ##Resultado
> ifelse(fc>ft,RN,RR) #Cuidado, aqui temos um teste unilateral a esquerda!
\end{Sinput}
\begin{Soutput}
[1] "Não rejeita-se H0 ao nível alpha=5% de significância"
\end{Soutput}
\begin{Sinput}
> ##Ou, equivalentemente:
> ifelse(pvalor > 0.05, RN, RR) 
\end{Sinput}
\begin{Soutput}
[1] "Não rejeita-se H0 ao nível alpha=5% de significância"
\end{Soutput}
\begin{Sinput}
> #H0:\muA=\muB
> #H1:\muA!=\muB
> (barxA <- 48)
\end{Sinput}
\begin{Soutput}
[1] 48
\end{Soutput}
\begin{Sinput}
> (barxB <- 52)
\end{Sinput}
\begin{Soutput}
[1] 52
\end{Soutput}
\begin{Sinput}
> (nA <- 15)
\end{Sinput}
\begin{Soutput}
[1] 15
\end{Soutput}
\begin{Sinput}
> (nB <- 12)
\end{Sinput}
\begin{Soutput}
[1] 12
\end{Soutput}
\begin{Sinput}
> df <- nA+nB-2
> Sc2 <- ((nA-1)*(dpA^2)+(nB-1)*(dpB^2))/(nA+nB-2)
> (tc <- (barxA-barxB)/(Sc2*sqrt((1/nA)+(1/nB))))
\end{Sinput}
\begin{Soutput}
[1] -0.06663197
\end{Soutput}
\begin{Sinput}
> (tt1 <- qt(0.025,df)) #Teste bilateral
\end{Sinput}
\begin{Soutput}
[1] -2.059539
\end{Soutput}
\begin{Sinput}
> (tt2 <- qt(0.975,df)) #Teste bilateral
\end{Sinput}
\begin{Soutput}
[1] 2.059539
\end{Soutput}
\begin{Sinput}
> (pvalor <- 2*(min(pt(tc,df),(1-pt(tc,df)))))
\end{Sinput}
\begin{Soutput}
[1] 0.9474047
\end{Soutput}
\begin{Sinput}
> RR <- "Rejeita-se H0 ao nível de 5% de significância"
> RN <- "Não rejeita-se H0 ao nível de 5% de significância"
> ##Resultado
> ifelse(abs(tc)>abs(tt),RR,RN)
\end{Sinput}
\begin{Soutput}
[1] "Não rejeita-se H0 ao nível de 5% de significância"
\end{Soutput}
\begin{Sinput}
> ##Ou, equivalentemente:
> ifelse(pvalor > 0.05, RN, RR)
\end{Sinput}
\begin{Soutput}
[1] "Não rejeita-se H0 ao nível de 5% de significância"
\end{Soutput}
\end{Schunk}

\item[21.~]Para investigar a influência da opção profissional sobre o salário inicial de recém-formados, investigaram-se dois grupos de profissionais: um de liberais em geral 
e outro de formandos em Administração de Empresas. Com os resultados abaixo, expressos em salários mínimos, quais seriam suas conclusões?

\begin{tabular}{ccccccccc}\\ \hline
Liberais & 6,6 & 10,3 & 10,8 & 12,9 & 9,2 & 12,3 & 7,0 &  \\ \hline
Administradores & 8,1 & 9,8 & 8,7 & 10,0 & 10,2 & 8,2 & 8,7 & 10,1 \\ \hline
\end{tabular}

\begin{Schunk}
\begin{Sinput}
> #H0:\sigma_{A}^{2}=\sigma_{B}^{2}
> #H1:\sigma_{A}^{2}!=\sigma_{B}^{2}
> (A <- Lib <- c(6.6 , 10.3 , 10.8 , 12.9 , 9.2 , 12.3 , 7.0))
\end{Sinput}
\begin{Soutput}
[1]  6.6 10.3 10.8 12.9  9.2 12.3  7.0
\end{Soutput}
\begin{Sinput}
> (B <- Adm <- c(8.1 , 9.8 , 8.7 , 10.0 , 10.2 , 8.2 , 8.7 , 10.1))
\end{Sinput}
\begin{Soutput}
[1]  8.1  9.8  8.7 10.0 10.2  8.2  8.7 10.1
\end{Soutput}
\begin{Sinput}
> (dpA <- sd(A))
\end{Sinput}
\begin{Soutput}
[1] 2.432909
\end{Soutput}
\begin{Sinput}
> (dpB <- sd(B))
\end{Sinput}
\begin{Soutput}
[1] 0.8876132
\end{Soutput}
\begin{Sinput}
> (nA <- length(A))
\end{Sinput}
\begin{Soutput}
[1] 7
\end{Soutput}
\begin{Sinput}
> (dfA <- nA-1)
\end{Sinput}
\begin{Soutput}
[1] 6
\end{Soutput}
\begin{Sinput}
> (nB <- length(B))
\end{Sinput}
\begin{Soutput}
[1] 8
\end{Soutput}
\begin{Sinput}
> (dfB <- nB-1)
\end{Sinput}
\begin{Soutput}
[1] 7
\end{Soutput}
\begin{Sinput}
> (alpha <- 0.05)
\end{Sinput}
\begin{Soutput}
[1] 0.05
\end{Soutput}
\begin{Sinput}
> (fc <- (dpA^2)/(dpB^2))
\end{Sinput}
\begin{Soutput}
[1] 7.512844
\end{Soutput}
\begin{Sinput}
> (ft1 <- qf(0.025,dfA,dfB, lower.tail=TRUE))
\end{Sinput}
\begin{Soutput}
[1] 0.1755781
\end{Soutput}
\begin{Sinput}
> (ft2 <- qf(0.975,dfA,dfB, lower.tail=TRUE))
\end{Sinput}
\begin{Soutput}
[1] 5.118597
\end{Soutput}
\begin{Sinput}
> (pvalor <-2*pf(fc,dfA,dfB,lower.tail=FALSE))
\end{Sinput}
\begin{Soutput}
[1] 0.01768275
\end{Soutput}
\begin{Sinput}
> (var.test(A,B,alternative = "two.sided"))
\end{Sinput}
\begin{Soutput}
	F test to compare two variances

data:  A and B
F = 7.5128, num df = 6, denom df = 7, p-value = 0.01768
alternative hypothesis: true ratio of variances is not equal to 1
95 percent confidence interval:
  1.467755 42.789180
sample estimates:
ratio of variances 
          7.512844 
\end{Soutput}
\begin{Sinput}
> RR <- "Rejeita-se H0 ao nível alpha=5% de significância"
> RN <- "Não rejeita-se H0 ao nível alpha=5% de significância"
> ##Resultado
> ifelse(fc>ft,RR,RN) #Cuidado, aqui temos um teste unilateral a esquerda!
\end{Sinput}
\begin{Soutput}
[1] "Rejeita-se H0 ao nível alpha=5% de significância"
\end{Soutput}
\begin{Sinput}
> ##Ou, equivalentemente:
> ifelse(pvalor > 0.05, RN, RR) 
\end{Sinput}
\begin{Soutput}
[1] "Rejeita-se H0 ao nível alpha=5% de significância"
\end{Soutput}
\begin{Sinput}
> #H0:\muA=\muB
> #H1:\muA!=\muB
> (barxA <- mean(A))
\end{Sinput}
\begin{Soutput}
[1] 9.871429
\end{Soutput}
\begin{Sinput}
> (barxB <- mean(B))
\end{Sinput}
\begin{Soutput}
[1] 9.225
\end{Soutput}
\begin{Sinput}
> #Variâncias distintas
> (df <- ((((dpA^2)/nA)+((dpB^2)/nB))^2)/(((((dpA^2)/nA)^2)/dfA)+((((dpB^2)/nB)^2)/dfB)))
\end{Sinput}
\begin{Soutput}
[1] 7.393037
\end{Soutput}
\begin{Sinput}
> (tc <- (barxA-barxB)/(sqrt(((dpA^2)/nA)+((dpB^2)/nB))))
\end{Sinput}
\begin{Soutput}
[1] 0.6653048
\end{Soutput}
\begin{Sinput}
> (tt1 <- qt(0.025,df)) #Teste bilateral
\end{Sinput}
\begin{Soutput}
[1] -2.339377
\end{Soutput}
\begin{Sinput}
> (tt2 <- qt(0.975,df)) #Teste bilateral
\end{Sinput}
\begin{Soutput}
[1] 2.339377
\end{Soutput}
\begin{Sinput}
> (pvalor <- 2*(min(pt(tc,df),(1-pt(tc,df)))))
\end{Sinput}
\begin{Soutput}
[1] 0.526061
\end{Soutput}
\begin{Sinput}
> (t.test(A,B,alternative = "two.sided"))
\end{Sinput}
\begin{Soutput}
	Welch Two Sample t-test

data:  A and B
t = 0.6653, df = 7.393, p-value = 0.5261
alternative hypothesis: true difference in means is not equal to 0
95 percent confidence interval:
 -1.626575  2.919433
sample estimates:
mean of x mean of y 
 9.871429  9.225000 
\end{Soutput}
\begin{Sinput}
> RR <- "Rejeita-se H0 ao nível de 5% de significância"
> RN <- "Não rejeita-se H0 ao nível de 5% de significância"
> ##Resultado
> ifelse(abs(tc)>abs(tt),RR,RN)
\end{Sinput}
\begin{Soutput}
[1] "Não rejeita-se H0 ao nível de 5% de significância"
\end{Soutput}
\begin{Sinput}
> ##Ou, equivalentemente:
> ifelse(pvalor > alpha, RN, RR)
\end{Sinput}
\begin{Soutput}
[1] "Não rejeita-se H0 ao nível de 5% de significância"
\end{Soutput}
\end{Schunk}

\item[22.~]Os dados abaixo referem-se a medidas de determinada variável em 19 pessoas antes e depois de uma cirurgia. Verifique se as medidas pré e pós-operatórias 
apresentam a mesma média. Que suposições você faria para resolver o problema?

\begin{tabular}{c|c|c|c|c|c} \hline
 Pessoas  & Pré  & Pós  & Pessoas & Pré  & Pós \\ \hline
 1        & 50,0 & 42,0 & 11      & 50,0 & 48,0      \\
 2        & 50,0 & 42,0 & 12      & 75,0 & 52,0 \\
 3        & 50,0 & 78,0 & 13      & 92,5 & 74,0 \\
 4        & 87,5 & 33,0 & 14      & 38,0 & 47,5 \\
 5        & 32,5 & 96,0 & 15      & 46,5 & 49,0 \\
 6        & 35,0 & 82,0 & 16      & 50,0 & 58,0 \\
 7        & 40,0 & 44,0 & 17      & 30,0 & 42,0 \\
 8        & 45,0 & 31,0 & 18      & 35,0 & 60,0 \\
 9        & 62,5 & 87,0 & 19      & 39,4 & 28,0 \\
 10       & 40,0 & 50,0 & 20      &   -  &  -   \\ \hline
\end{tabular}

\begin{Schunk}
\begin{Sinput}
> (A <- Pre <- c(50.0,50.0,50.0,87.5,32.5,35.0,40.0,45.0,62.5,40.0,50.0,
+                75.0,92.5,38.0,46.5,50.0,30.0,35.0,39.4))
\end{Sinput}
\begin{Soutput}
 [1] 50.0 50.0 50.0 87.5 32.5 35.0 40.0 45.0 62.5 40.0 50.0 75.0 92.5 38.0 46.5
[16] 50.0 30.0 35.0 39.4
\end{Soutput}
\begin{Sinput}
> (B <- Pos <- c(42.0,42.0,78.0,33.0,96.0,82.0,44.0,31.0,87.0,50.0,48.0,
+                52.0,74.0,47.5,49.0,58.0,42.0,60.0,28.0))
\end{Sinput}
\begin{Soutput}
 [1] 42.0 42.0 78.0 33.0 96.0 82.0 44.0 31.0 87.0 50.0 48.0 52.0 74.0 47.5 49.0
[16] 58.0 42.0 60.0 28.0
\end{Soutput}
\begin{Sinput}
> #H0:\muA=\muB (d=\muA-\muB=0)
> #H1:\muA!=\muB (d=\muA-\muB!=0)
> (d <- A-B)
\end{Sinput}
\begin{Soutput}
 [1]   8.0   8.0 -28.0  54.5 -63.5 -47.0  -4.0  14.0 -24.5 -10.0   2.0  23.0
[13]  18.5  -9.5  -2.5  -8.0 -12.0 -25.0  11.4
\end{Soutput}
\begin{Sinput}
> (n <- length(d))
\end{Sinput}
\begin{Soutput}
[1] 19
\end{Soutput}
\begin{Sinput}
> (df <- n-1)
\end{Sinput}
\begin{Soutput}
[1] 18
\end{Soutput}
\begin{Sinput}
> (bard <- mean(d))
\end{Sinput}
\begin{Soutput}
[1] -4.978947
\end{Soutput}
\begin{Sinput}
> (Sd <- sd(d))
\end{Sinput}
\begin{Soutput}
[1] 26.35174
\end{Soutput}
\begin{Sinput}
> (tc <- (bard)/(Sd/sqrt(n)))
\end{Sinput}
\begin{Soutput}
[1] -0.8235787
\end{Soutput}
\begin{Sinput}
> alpha <- 0.05
> (tt1 <- qt(0.025,df)) #Teste bilateral
\end{Sinput}
\begin{Soutput}
[1] -2.100922
\end{Soutput}
\begin{Sinput}
> (tt2 <- qt(0.975,df)) #Teste bilateral
\end{Sinput}
\begin{Soutput}
[1] 2.100922
\end{Soutput}
\begin{Sinput}
> (pvalor <- 2*(min(pt(tc,df),(1-pt(tc,df)))))
\end{Sinput}
\begin{Soutput}
[1] 0.4209576
\end{Soutput}
\begin{Sinput}
> t.test(d,alternative = "two.sided")
\end{Sinput}
\begin{Soutput}
	One Sample t-test

data:  d
t = -0.82358, df = 18, p-value = 0.421
alternative hypothesis: true mean is not equal to 0
95 percent confidence interval:
 -17.680077   7.722183
sample estimates:
mean of x 
-4.978947 
\end{Soutput}
\begin{Sinput}
> RR <- "Rejeita-se H0 ao nível de 5% de significância"
> RN <- "Não rejeita-se H0 ao nível de 5% de significância"
> ##Resultado
> ifelse(abs(tc)>abs(tt),RR,RN)
\end{Sinput}
\begin{Soutput}
[1] "Não rejeita-se H0 ao nível de 5% de significância"
\end{Soutput}
\begin{Sinput}
> ##Ou, equivalentemente:
> ifelse(pvalor > alpha, RN, RR)
\end{Sinput}
\begin{Soutput}
[1] "Não rejeita-se H0 ao nível de 5% de significância"
\end{Soutput}
\end{Schunk}
\end{ManualExercise}
\end{Exercise}

\end{document}
