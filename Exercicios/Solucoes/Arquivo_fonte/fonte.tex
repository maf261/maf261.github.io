\documentclass{article}

\usepackage[brazil]{babel}
\usepackage[utf8]{inputenc}
%\UseRawInputEncoding
\usepackage[T1]{fontenc}
\usepackage{Sweave}
\usepackage{animate}
\usepackage{amsbsy}
\usepackage{amsfonts}
\usepackage{amsmath}
\usepackage{amssymb}
\usepackage{amsthm}
\usepackage[toc,page,title,titletoc]{appendix}
\usepackage[fixlanguage]{babelbib}
%\usepackage[pdftex]{color}
\usepackage{dsfont}
\usepackage{esvect}
\usepackage[labelfont=bf]{caption}
\usepackage{float}
\usepackage[Glenn]{fncychap}%Sonny %Conny %Lenny %Glenn %Renje %Bjarne %Bjornstrup
%\usepackage{geometry, calc, color, setspace}%
%\geometry{a4paper, headsep=1.0cm, footskip=1cm, lmargin=3cm, rmargin=2cm, tmargin=3cm, bmargin=2cm}
\usepackage{graphicx}
\usepackage{indentfirst}%Para indentar os parágrafos automáticamente
\usepackage{lipsum}
\usepackage{longtable}
\usepackage{mathtools}
\usepackage{listings}%Inserir codigo do R no latex
\usepackage{multirow}
\usepackage{multicol}
\usepackage{ifthen}
\newboolean{firstanswerofthechapter}  
\usepackage{natbib}
\bibliographystyle{abbrvnat3}
\usepackage[figuresright]{rotating}
\usepackage{spalign}
%\usepackage{pgfpages}
\usepackage{pgfplots}
\usepackage{tikz}
\usepackage{tasks}
\usepackage{color, colortbl}
\usepackage{xcolor}
\colorlet{lightcyan}{cyan!40!white}
\usepackage{chngcntr}
\usepackage{stackengine}
\usepackage{ragged2e}%para justificar o texto dentro de algum ambiente
\definecolor{Gray}{gray}{0.9}
\definecolor{LightCyan}{rgb}{0.88,1,1}


\usepackage[all]{xy}
\usepackage{hyperref,bookmark}
\hypersetup{
  colorlinks=true,
  linkcolor=blue,
  citecolor=red,
  filecolor=blue,
  urlcolor=blue,
}
\newlength{\longestlabel}
\settowidth{\longestlabel}{\bfseries viii.}

\setcounter{secnumdepth}{0} \setlength{\topmargin}{0cm}
\setlength{\headsep}{-0.3cm} \setlength{\textwidth}{17.5cm}
\setlength{\textheight}{23cm} \setlength{\oddsidemargin}{-0.8cm}
\setlength{\evensidemargin}{0cm} \setlength{\footskip}{-1.5cm}

\usepackage[lastexercise,answerdelayed]{exercise}
\renewcommand{\ExerciseName}{Exercícios}
\renewcommand{\ExerciseHeader}{\noindent\def\stackalignment{l}% code from https://tex.stackexchange.com/a/195118/101651
    \stackunder[0pt]{\colorbox{cyan}{\textcolor{white}{\textbf{\large\ExerciseName}}}}{\textcolor{lightcyan}{\rule{\linewidth}{2pt}}}\medskip}


\begin{document}
\Sconcordance{concordance:fonte.tex:fonte.Rnw:%
1 188 1}


\section{Teste z - Unilateral a Esquerda}

% \begin{align*}
% H_{0}: \mu&=AQUI \\ 
% H_{1}: \mu&<AQUI\quad \textrm{(unilateral)}
% \end{align*}
% 
% Dados:
% 
% $n=AQUI;\quad \bar{x}=AQUI;\quad \sigma=AQUI;\quad \alpha=AQUI\% \rightarrow z_{t}=AQUI$
% 
% $$z_{cal}=\dfrac{\bar{x}-\mu_{0}}{\dfrac{\sigma}{\sqrt{n}}}=\dfrac{AQUI-AQUI}{AQUI/\sqrt{AQUI}}=AQUI$$
% $$p-valor=AQUI$$
% \begin{center}
% \setkeys{Gin}{width=0.5\linewidth}
% <<echo=FALSE,fig.pos="h",fig=TRUE>>=
% zt <- qnorm(AQUI)
% zc <- (AQUI-AQUI)/(AQUI/sqrt(AQUI))
% pvalor <- pnorm(zc)
% curve(dnorm(x), from=-4.5, to=4.5, xlab="", ylab="")
% polygon(cbind(c(zt,seq(zt,-4.5, l=100),-4.5), 
%               c(0, dnorm(seq(zt, -4.5, l=100)), 
%                 dnorm(-4.5))), 
%         col="lightgray")
% abline(v=zt, lty=2)
% arrows(zc, 0.1, zc, 0)
% zt <- format(zt,digits = 3)
% Zt <- bquote(bold(z[t] == .(zt)))
% zc <- format(zc,digits = 3)
% Zc <- bquote(bold(z[c] == .(zc)))
% text(zt, 0.1, Zt, pos=4)
% text(zt, 0.2, "RNRH0", pos=4)
% text(zc, 0.12, Zc, pos=3)
% text(zc, 0.2, "RRH0", pos=3)
% @
% \end{center}
% 
% Decisão: %Como $|z_{cal}|>|z_{tab}|$ rejeita-se $H_{0}$ ao nível $\alpha=AQUI\%$ de significância.
% %Como $|z_{cal}|>|z_{tab}|$ não rejeita-se $H_{0}$ ao nível $\alpha=AQUI\%$ de significância.
% Comandos em R para soluções:
% 
% <<>>=
% AQUI
% RR <- "Rejeita-se H0 ao nível de 5% de significância"
% RN <- "Não rejeita-se H0 ao nível de 5% de significância"
% ##Resultado
% ifelse(zc>zt,RR,RN)
% ##Ou, equivalentemente:
% ifelse(pvalor > AQUI, RN, RR)
% ## estimativa pontual
% (mu.est <- AQUI)
% ## estimativa intervalar (95%)
% (IC.mu <- mu.est + qnorm(c(0.025, 0.975)) * AQUI/sqrt(AQUI))
% @

\section{Teste z - Unilateral a Direita}

% \begin{align*}
% H_{0}: \mu&=AQUI \\ 
% H_{1}: \mu&>AQUI\quad \textrm{(unilateral)}
% \end{align*}
% 
% Dados:
% 
% $n=AQUI;\quad \bar{x}=AQUI;\quad \sigma=AQUI;\quad \alpha=AQUI\% \rightarrow z_{t}=AQUI$
% 
% $$z_{cal}=\dfrac{\bar{x}-\mu_{0}}{\dfrac{\sigma}{\sqrt{n}}}=\dfrac{AQUI-AQUI}{AQUI/\sqrt{AQUI}}=AQUI$$
% $$p-valor=AQUI$$
% \begin{center}
% \setkeys{Gin}{width=0.5\linewidth}
% <<echo=FALSE,fig.pos="h",fig=TRUE>>=
% zt <- qnorm(AQUI)
% zc <- (AQUI-AQUI)/(AQUI/sqrt(AQUI))
% pvalor <- 1-pnorm(zc)
% curve(dnorm(x), from=-4.5, to=4.5, xlab="", ylab="")
% polygon(cbind(c(zt,seq(zt,4.5, l=100),4.5), 
%               c(0, dnorm(seq(zt, 4.5, l=100)), 
%                 dnorm(4.5))), 
%         col="lightgray")
% abline(v=zt, lty=2)
% arrows(zc, 0.1, zc, 0)
% zt <- format(zt,digits = 3)
% Zt <- bquote(bold(z[t] == .(zt)))
% zc <- format(zc,digits = 3)
% Zc <- bquote(bold(z[c] == .(zc)))
% text(zt, 0.1, Zt, pos=2)
% text(zt, 0.2, "RNRH0", pos=2)
% text(zc, 0.12, Zc, pos=4)
% text(zc, 0.2, "RRH0", pos=4)
% @
% \end{center}
% 
% Decisão: Como %$|z_{cal}|>|z_{tab}|$ rejeita-se $H_{0}$ ao nível $\alpha=AQUI\%$ de significância.
% %$|z_{cal}|<|z_{tab}|$ não rejeita-se $H_{0}$ ao nível $\alpha=AQUI\%$ de significância.
% Comandos em R para soluções:
% 
% <<>>=
% AQUI
% RR <- "Rejeita-se H0 ao nível de 5% de significância"
% RN <- "Não rejeita-se H0 ao nível de 5% de significância"
% ##Resultado
% ifelse(zc>zt,RR,RN)
% ##Ou, equivalentemente:
% ifelse(pvalor > AQUI, RN, RR)
% ## estimativa pontual
% (mu.est <- AQUI)
% ## estimativa intervalar (95%)
% (IC.mu <- mu.est + qnorm(c(0.025, 0.975)) * AQUI/sqrt(AQUI))
% @


\end{document}
