\documentclass[14pt,aspectratio=1610]{beamer}

\usepackage[brazil]{babel}
\usepackage[utf8]{inputenc}
\usepackage[T1]{fontenc}
\mathchardef\hyphenmathcode=\mathcode`\-
\usepackage{listings}
%\usepackage{xr-hyper}

\usepackage{sansmathaccent}
\pdfmapfile{+sansmathaccent.map}
\usepackage{amsbsy}
\usepackage{amsfonts}
\usepackage{amsmath}
\usepackage{amssymb}
\usepackage{amsthm}
\usepackage[toc,page,title,titletoc]{appendix}
\usepackage[fixlanguage]{babelbib}
%\usepackage[pdftex]{color}
\usepackage{dsfont}
\usepackage{esvect}
\usepackage[labelfont=bf]{caption}
\usepackage{float}
\usepackage[Glenn]{fncychap}%Sonny %Conny %Lenny %Glenn %Renje %Bjarne %Bjornstrup
%\usepackage{geometry, calc, color, setspace}%
%\geometry{a4paper, headsep=1.0cm, footskip=1cm, lmargin=3cm, rmargin=2cm, tmargin=3cm, bmargin=2cm}
\usepackage{graphicx}
\usepackage{indentfirst}%Para indentar os parágrafos automáticamente
\usepackage{lipsum}
\usepackage{longtable}
\usepackage{mathtools}
\usepackage{multirow}
\usepackage{multicol}
\usepackage{natbib}
\bibliographystyle{abbrvnat3}
\usepackage[figuresright]{rotating}
\usepackage{spalign}
%\usepackage{pgfpages}
\usepackage{pgfplots}
\usepackage{tikz}
\usepackage{color, colortbl}
\usepackage{ragged2e}%para justificar o texto dentro de algum ambiente
\definecolor{Gray}{gray}{0.9}
\definecolor{LightCyan}{rgb}{0.88,1,1}


\usepackage[all]{xy}
\usepackage{hyperref,bookmark}
\hypersetup{
  colorlinks=true,
  linkcolor=blue,
  citecolor=red,
  filecolor=blue,
  urlcolor=blue,
}


%\setcitestyle{authoryear, open={(},close={)}}
%\usepackage{pgf}
%\usepackage[small,bf,singlelinecheck=off]{caption}
%\usepackage[figuresright]{rotating}

%\usepackage[font=Times,timeinterval=10, timeduration=2.0, timedeath=0, fillcolorwarningsecond=white!60!yellow,
%timewarningfirst=50,timewarningsecond=80,resetatpages=2]{tdclock}
\usetheme{Madrid}
\usecolortheme[RGB={193,0,0}]{structure}

%\setbeamertemplate{footline}[frame number]
%\setbeamertemplate{footline}[text line]{%
%  \parbox{\linewidth}{\vspace*{-8pt}\hfill\date{}\hfill\insertshortauthor\hfill\insertpagenumber}}
\beamertemplatenavigationsymbolsempty
\renewcommand{\vec}[1]{\mbox{\boldmath$#1$}}
\newtheorem{Teorema}{Teorema}
\newtheorem{Proposicao}{Proposição}
\newtheorem{Definicao}{Definição}
\newtheorem{Corolario}{Corolário}
\newtheorem{Demonstracao}{Demonstração}
\newcommand{\bx}{\ensuremath{\bar{x}}}
\newcommand{\Ho}{\ensuremath{H_{0}}}
\newcommand{\Hi}{\ensuremath{H_{1}}}


\apptocmd{\frame}{}{\justifying}{} % Allow optional arguments after frame.

\title{MAF 261 - Estatística Experimental}
\author{Prof. Fernando de Souza Bastos}
\institute{Instituto de Ciências Exatas e Tecnológicas\texorpdfstring{\\ Universidade Federal de Viçosa}{}\texorpdfstring{\\ Campus UFV - Florestal}{}}
\date{14/08/2018}
\newcommand\mytext{Aula 3}
\newcommand\mytextt{Fernando de Souza Bastos}
\makeatletter
\setbeamertemplate{footline}
{
  \leavevmode%
  \hbox{%
  \begin{beamercolorbox}[wd=.333333\paperwidth,ht=2.25ex,dp=1ex,center]{author in head/foot}%
    \usebeamerfont{author in head/foot}\mytext
  \end{beamercolorbox}%
  \begin{beamercolorbox}[wd=.333333\paperwidth,ht=2.25ex,dp=1ex,center]{title in head/foot}%
    \usebeamerfont{title in head/foot}\mytextt
  \end{beamercolorbox}%
  \begin{beamercolorbox}[wd=.333333\paperwidth,ht=2.25ex,dp=1ex,right]{date in head/foot}%
    \usebeamerfont{date in head/foot}\insertshortdate{}\hspace*{2em}
    \insertframenumber{} / \inserttotalframenumber\hspace*{2ex} 
  \end{beamercolorbox}}%
  \vskip0pt%
}
\makeatother


\providecommand{\arcsin}{} \renewcommand{\arcsin}{\hspace{2pt}\textrm{arcsen}}
\providecommand{\sin}{} \renewcommand{\sin}{\hspace{2pt}\textrm{sen}}
%\newtheorem{Teorema}{Teorema}
%\newtheorem{Proposicao}{Proposição}
%\newtheorem{Definicao}{Definição}
%\newtheorem{Corolario}{Corolário}
%\newtheorem{Demonstracao}{Demonstração}

% Layout da pagina
\hypersetup{pdfpagelayout=SinglePage}
\begin{document}

\pgfmathdeclarefunction{gauss}{2}{%
  \pgfmathparse{1/(#2*sqrt(2*pi))*exp(-((x-#1)^2)/(2*#2^2))}%
}


\frame{\titlepage}

\begin{frame}{}
\frametitle{\bf Sumário}
\tableofcontents
\end{frame}

\section{Teste de Hipóteses}
\begin{frame}{}
\frametitle{Testes de hipóteses}
\begin{block}{Exemplo prático (\cite{montgomery2016}):}
\justifying
Suponha que um engenheiro esteja projetando um sistema de escape da tripulação de uma aeronave, que consiste em um assento de ejeção e um motor de foguete 
que energiza o assento. O motor de foguete contém um propelente. Para o assento de ejeção funcionar apropriadamente, o propelente deve ter uma taxa mínima de 
queima de $50$ cm/s. Se a taxa de queima for muito baixa, o assento de ejeção poderá não funcionar apropriadamente, levando a uma ejeção não segura. Taxas maiores 
de queima podem implicar instabilidade no propelente ou um assento de ejeção muito potente, levando outra vez a insegurança da injeção. Dessa maneira, a questão 
prática de engenharia que tem de ser respondida é: a taxa média de queima do propelente é igual a $50$ cm/s ou é igual a algum outro valor 
(maior ou menor)?
\end{block}
\end{frame}

\begin{frame}{}
\frametitle{Testes de hipóteses}
\begin{block}{Hipótese Estatística:}
Uma hipótese estatística é uma afirmação sobre os parâmetros de uma ou mais populações.
 \end{block}
\end{frame}

\begin{frame}{}
\frametitle{Testes de hipóteses}
\begin{block}{}
\justifying
Considere o sistema de escape da tripulação descrito no exemplo anterior. Suponha que estejamos interessados na taxa de queima do propelente sólido. Agora, a taxa de 
queima é uma variável aleatória que pode ser descrita por uma distribuição de probabilidades. Suponha que nosso interesse esteja focado na taxa média de queima 
(um parâmetro dessa distribuição). Especificamente, estamos interessados em decidir se a taxa média de queima é ou não 50 centímetros por segundo. Podemos 
expressar isso formalmente como:
\begin{align}\label{H0eH1}
\centering
H_{0}: \mu&=50 cm/s\\
\nonumber H_{1}:\mu &\neq 50 cm/s
\end{align}
 \end{block}
\end{frame}
%\begin{frame}{}
%\frametitle{Testes de hipóteses}
%\begin{block}{}
%Em algumas situações, podemos desejar formular uma hipótese alternativa unilateral, como em:
%\begin{align}
%\centering
%\nonumber  H_{0}: \mu=50 cm/s  &&                  &&H_{0}: \mu&=50 cm/s\\
                    %H_{1}:\mu> 50 cm/s  &&\textrm{ou}&&H_{1}:\mu &< 50 cm/s
%\end{align}
%\end{block}
%\end{frame}
\begin{frame}{}
\frametitle{Testes de hipóteses}
\begin{block}{}
\justifying
A afirmação $H_{0}: \mu=50$ centímetros por segundo na Equação \ref{H0eH1} é chamada de hipótese nula, e a afirmação $H_{1}:\mu \neq 50$ centímetros por segundo 
é chamada de hipótese alternativa. Uma vez que a hipótese alternativa especifica valores de $\mu$ que poderiam ser maiores ou menores do que $50$ centímetros por 
segundo, ela é chamada de hipótese alternativa bilateral.
\end{block}
\pause
\begin{block}{}
Em algumas situações, podemos desejar formular uma hipótese alternativa unilateral, como em:
\begin{flalign}
\begin{aligned} 
	\begin{cases}
H_{0}: \mu=50 cm/s\\
H_{1}:\mu> 50 cm/s
\end{cases}
\end{aligned}
\quad\textrm{ou}\quad
\begin{aligned}
\begin{cases}
H_{0}: \mu=50 cm/s\\
H_{1}:\mu< 50 cm/s
\end{cases} \\
\end{aligned}
\end{flalign}
 \end{block}
\end{frame}

\begin{frame}{}
\frametitle{Testes de hipóteses}

\begin{block}{}
\justifying
Sempre estabeleceremos a hipótese nula como uma reivindicação de igualdade. Entretanto, quando a hipótese alternativa for estabelecida com o sinal $<,$ a reivindicação 
implícita na hipótese nula será $\geq$ e quando a hipótese alternativa for estabelecida com o sinal $>,$ a reivindicação implícita na hipótese nula será $\leq.$
\end{block}
\end{frame}

\begin{frame}{}
\frametitle{Testes de hipóteses}
\begin{block}{}
\justifying
É importante lembrar que hipóteses são sempre afirmações sobre a população ou distribuição sob estudo, não afirmações sobre a amostra. O valor do parâmetro 
especificado da população na hipótese nula (50 centímetros por segundo no exemplo anterior) é geralmente determinado em uma das três maneiras. 
\begin{enumerate}
\item experiência passada, conhecimento do processo ou experimentos prévios. O objetivo nesse caso é determinar se o valor do parâmetro variou;\pause
\item alguma teoria ou modelo relativo ao processo sob estudo. Aqui, o objetivo do teste é verificar a teoria ou modelo;\pause
\item considerações externas, tais como projeto ou especificações de engenharia, ou a partir de obrigações contratuais. Nessa situação, o objetivo usual é avaliar a 
correção das especificações.
\end{enumerate}
 \end{block}
\end{frame}

\begin{frame}{}
\frametitle{Testes de hipóteses}
\begin{block}{}
\justifying
\textbf{Teste de hipóteses se apoiam no uso de informações de uma amostra aleatória proveniente da população de interesse}. É importante ressaltar que a verdade ou 
falsidade de uma hipótese particular pode nunca ser conhecida com certeza, a menos que possamos examinar a população inteira. Testar uma hipótese envolve: 
\begin{itemize}
\item considerar uma amostra aleatória; \pause
\item computar uma estatística de teste a partir de dados amostrais \pause
\item e então usar a estatística de teste para tomar uma decisão a respeito da hipótese nula.
\end{itemize}
\end{block}
\end{frame}

\begin{frame}{}
\frametitle{ Testes de Hipóteses Estatísticas}
\begin{block}{}
\justifying
A hipótese nula corresponde à taxa média de queima ser igual a 50 centímetros por segundo e a alternativa corresponde a essa taxa não ser igual a $50$ centímetros 
por segundo. Ou seja, desejamos testar
\begin{align*}
\centering
H_{0}: \mu=50 cm/s &&\textrm{contra}&& H_{1}:\mu \neq 50 cm/s
\end{align*}
Suponha que uma amostra de $n = 10$ espécimes seja testada e que a taxa média $\bar{x}$ seja observada. A média amostral é uma estimativa de $\mu.$ Um valor de 
$\bar{x}$ que caia próximo a $\mu = 50$ cm/s é uma evidência de que $\mu$ é realmente $50$ cm/s. Por outro lado, 
uma média amostral que seja consideravelmente diferente de $50$ cm/s evidencia a validade da hipótese alternativa $H_{1}.$ Assim, a média amostral 
é a estatística de teste nesse caso.
\end{block}
\end{frame}


\begin{frame}{}
\frametitle{Testes de hipóteses}
\begin{block}{}
\justifying
A média amostral pode assumir muitos valores diferentes. Suponha que se $48,5 \leq \bar{x}\leq 51,5,$ não rejeitaremos a hipótese nula $H_{0}:\mu = 50$ e se 
$\bar{x} < 48,5$ ou $\bar{x} > 51,5,$ rejeitaremos a hipótese nula em favor da hipótese alternativa $H_{1}:\mu \neq 50.$ Isso é ilustrado na Figura abaixo:
 \end{block}\pause
\begin{block}{}
\begin{figure}
\centering
\begin{tikzpicture}[xscale=1.5, yscale=1.5, <->=triangle 45]
\draw [>=stealth] (-5,0) -- (5,0);
\node [below right] at (4.7,0) {$\bar{x}$} ;
\draw [-] (-2,0) -- (-2,1.5) ;
\node [below] at (-2,0) {$48.5$} ;
\node [above] at (-3.5,0.7) {Rejeita $H_{0}$};
\node [above] at (-3.5,0.2) {$\mu\neq 50$ cm/s};
\draw [-] (0,0) -- (0,0.1) ;
\node [below] at (0,0) {$50$} ;
\node [above] at (0,0.7) {Falha em rejeitar $H_{0}$};
\node [above] at (0,0.2) {$\mu= 50$ cm/s};
\draw [-] (2,0) -- (2,1.5) ;
\node [below] at (2,0) {$51.5$} ;
\node [above] at (3.5,0.7) {Rejeita $H_{0}$};
\node [above] at (3.5,0.2) {$\mu\neq 50$ cm/s};
\end{tikzpicture}
\caption{Critérios de decisão para testar $H_{0}:\mu = 50$ cm/s versus $H_{1}: \mu \neq 50$ cm/s.} \label{fig:M1}
\end{figure}
 \end{block}
\end{frame}

\begin{frame}{}
\frametitle{Testes de hipóteses}
\begin{block}{}
\justifying
Os valores de $\bar{x}$ que forem menores do que $48,5$ e maiores do que $51,5$ constituem a \textbf{região crítica} para o teste, enquanto todos os valores que estejam 
no intervalo $48,5 \leq \bar{x}\leq 51,5$ formam uma região para a qual falharemos em rejeitar a hipótese nula. Por convenção, ela geralmente é chamada de 
\textbf{região de não rejeição}. Os limites entre as regiões críticas e a região de aceitação são chamados de valores críticos. 
 \end{block}
\end{frame}

\begin{frame}{}
\frametitle{Testes de hipóteses}
\begin{block}{}
\justifying
Em nosso exemplo, os valores críticos são $48,5$ e $51,5.$ É comum estabelecer conclusões relativas à hipótese nula $H_{0}.$ Logo, rejeitaremos $H_{0}$ em favor 
de $H_{1}$, se a estatística de teste cair na região crítica e falhamos em rejeitar $H_{0}$ por sua vez se a estatística de teste cair na região de aceitação.
 \end{block}
\end{frame}

\section{Nível de significância}
\begin{frame}{}
\frametitle{Testes de hipóteses}
\begin{block}{}
\justifying
Esse procedimento pode levar a duas conclusões erradas. Por exemplo, a taxa média verdadeira de queima do propelente poderia ser igual a 50 centímetros por segundo. 
Entretanto, para as amostras de propelente, selecionados aleatoriamente, que são testados, poderíamos observar um valor de estatística de teste $\bar{x}$ que 
caísse na região crítica. Rejeitaríamos então a hipótese nula $H_{0}$ em favor da alternativa $H_{1},$ quando, de fato, $H_{0}$ seria realmente verdadeira. Esse tipo de 
conclusão errada é chamado de \textbf{erro tipo I}.
 \end{block}
\pause
\begin{block}{Erro Tipo I}
\begin{tikzpicture}
\node[draw,align=center, fill=gray!30] at (0,-1) {A rejeição da hipótese nula $H_{0}$ quando ela for verdadeira é definida como\\ \textbf{erro tipo I}.};
\end{tikzpicture}

\end{block}
\end{frame}

\begin{frame}{}
\frametitle{Testes de hipóteses}
\begin{block}{}
\justifying
Agora, suponha que a taxa média verdadeira de queima seja diferente de 50 centímetros por segundo, mesmo que a média amostral $\bar{x}$ caia na região de 
aceitação. Nesse caso, falharíamos em rejeitar $H_{0}$, quando ela fosse falsa. Esse tipo de conclusão errada é chamado de \textbf{erro tipo II}.
\end{block}
\pause
\begin{block}{Erro Tipo II}
\begin{tikzpicture}
\node[draw,align=center, fill=gray!30] at (0,-1) {A falha em rejeitar a hipótese nula, quando ela é falsa, é definida como\\ \textbf{erro tipo II}.};
\end{tikzpicture}
\end{block}
\end{frame}

\begin{frame}{}
\frametitle{Testes de hipóteses}
\begin{block}{}
\justifying
Assim, testando qualquer hipótese estatística, quatro situações diferentes determinam se a decisão final está correta ou errada. Pelo fato de a nossa decisão estar 
baseada em variáveis aleatórias, probabilidades podem ser associadas aos erros tipo I e tipo II. A probabilidade de cometer o erro tipo I é denotada pela letra grega 
$\alpha$.
 \end{block}

\begin{block}{}
\begin{center}
\begin{table}[]
\begin{tabular}{c|c|c}
&&\\
                          \textbf{Decisão}                   &\textbf{$H_{0}\,$ é verdadeira}                         &\textbf{$H_{0}$ é falsa}\\ \hline
 \multirow{2}{*}{\textbf{Não rejeita $H_{0}$}}&Correta                                                               &Erro Tipo II\\
                                                                        &\textbf{Probabilidade $=\left(1-\alpha \right)$} &\textbf{Probabilidade $=\beta$}\\ \hline
\multirow{2}{*}{\textbf{Rejeita $H_{0}$}}       &Erro Tipo I                                                          &  Correta\\
                                                                        &\textbf{Nível de significância $\alpha$}            &\textbf{Poder $=\left( 1-\beta\right)$}\\ \hline
\end{tabular}
\end{table}
\end{center}
\end{block}

\end{frame}

\begin{frame}{}
\frametitle{Testes de hipóteses}
\begin{block}{}
\justifying
A probabilidade do erro tipo I é chamada de \textbf{nível de significância}, ou \textbf{erro $\alpha$}, ou \textbf{tamanho do teste}. No exemplo da taxa de queima de 
propelente, um \textbf{erro tipo I} ocorrerá quando $ \bx> 51,5$ ou $\bx < 48,5,$ quando a taxa média verdadeira de queima do propelente for $\mu = 50$ cm/s. 
\end{block}
\end{frame}

\begin{frame}{}
\frametitle{Testes de hipóteses}
\begin{block}{}
\justifying
Suponha que o desvio-padrão da taxa de queima seja $\sigma = 2,5$ centímetros por segundo e que a taxa de queima tenha uma distribuição para a qual as condições do 
\textbf{teorema central do limite} se aplicam; logo, a distribuição da média amostral é aproximadamente normal, com média $\mu = 50$ e desvio-padrão 
$\dfrac{\sigma}{\sqrt{n}}=\dfrac{2.5}{\sqrt{10}}=0.79$.
\end{block}
\end{frame}

\begin{frame}{}
\frametitle{Testes de hipóteses}
\begin{block}{}
A probabilidade de cometer o \textbf{erro tipo I} (ou o nível de significância de nosso teste) é igual à soma das 
áreas sombreadas nas extremidades da distribuição normal na Figura abaixo:
\end{block}
\vspace{-0.5cm}
\begin{figure}
\centering
\begin{tikzpicture}[xscale=1.5, yscale=7, declare function={stdnorm(\x) = 1/(sqrt(2*pi))*exp(-0.5*(pow(\x,2)));}]
\fill[gray!30] (-2.5,0) -- plot [domain=-2.5:-3/2, samples=50] (\x, {stdnorm(\x)}) -- (-3/2,0) -- cycle;
\fill[gray!30] (3/2,0) -- plot [domain=3/2:5/2, samples=50] (\x, {stdnorm(\x)}) -- (5/2,0) -- cycle;
\draw [thick, domain=-2.5:2.5, samples=50] plot (\x, {stdnorm(\x)});
\draw [->] (-3,0) -- (3,0) ;
\node [below right] at (3,0) {$\bar{x}$} ;
\draw [dashed] (0,0) -- (0,{stdnorm(0)}) ;
\draw [dashed] (-3/2,0) -- (-3/2,{stdnorm(-3/2)}) ;
\draw [dashed] (3/2,0) -- (3/2,{stdnorm(3/2)}) ;
\node [below] at (0,0) {$50$};
\node [below] at (-3/2,0) {$48.5$};
\node [below] at (3/2,0) {$51.5$};
\node [above] at (-3,0.1) {\small{$\alpha/2=0.0287$}};
\node [above] at (3,0.1) {\small{$\alpha/2=0.0287$}};
%\draw[->] (-2.7,0.15) .. controls (.-2,.2) .. (-1.9, 0.03);
\draw[->] (-2.2,0.15) to [out=20,in=90] (-1.9,0.02);
\draw[->] (2.2,0.15) to [out=160,in=90] (1.9,0.02);
%\draw [->,thick] (2.7,0.15) to [out=120,in=0] (2.3,0.3)
%to [out=0,in=90] (1.9,0.03);
%\draw (0,0) .. controls (0,4) and (4,0) .. (4,4)
%\draw[->] ( 3,0.15) .. controls (. 30,.2) .. (1.9, 0.03);
%\node at (1.8,{stdnorm(2.3)}) {\small{$\alpha/2$}};
%\node at (-1.8,{stdnorm(2.3)}){\small{$\alpha/2$}};
\end{tikzpicture}
\caption{Região crítica para $\Ho: \mu = 50$ versus $\Hi: \mu \neq 50$ e $n = 10$}
\end{figure}
\vspace{-0.5cm}
\pause
\begin{tikzpicture}
\node[draw,align=center, fill=gray!30] at (0,-1) {$\alpha=P(\bar{X}<48.5\quad \textrm{quando}\quad \mu=50)+P(\bar{X}>51.5\quad \textrm{quando}\quad \mu=50)$.};
\end{tikzpicture}
%\begin{block}{}
%$$\alpha=P(\bar{X}<48.5\quad \textrm{quando}\quad \mu=50)+P(\bar{X}>51.5\quad \textrm{quando}\quad \mu=50)$$
%\end{block}
\end{frame}


\begin{frame}{}
\frametitle{}
\begin{block}{}
\justifying
Os valores de z que correspondem aos valores críticos $48,5$ e $51,5$ são

\begin{align*}
z_{1}=\dfrac{\bx-\mu}{\dfrac{\sigma}{\sqrt{n}}}=\dfrac{48.5-50}{0.79}=-1.9\quad \textrm{e}\quad z_{2}=\dfrac{\bx-\mu}{\dfrac{\sigma}{\sqrt{n}}}=\dfrac{51.5-50}{0.79}=1.9
\end{align*}
Logo,

$$\alpha=P(z<-1.90)+P(z>1.90)=0.0287+0.0287=0.0574$$

Essa é a probabilidade do erro tipo I. Isso implica que $5,74\%$ de todas as amostras aleatórias conduziriam à rejeição da hipótese $\Ho: \mu = 50$ cm/s, quando a 
taxa média verdadeira de queima fosse realmente 50 centímetros por segundo.Da inspeção da Figura anterior, notamos que podemos reduzir $\alpha$ alargando a 
região de aceitação.
\end{block}
\end{frame}

\begin{frame}{}
\frametitle{}
\begin{block}{}
\justifying
Por exemplo, se considerarmos os valores críticos $48$ e $52,$ o valor de $\alpha$ será:
\begin{align*}
\alpha&=P\Biggl(z<-\dfrac{48-50}{0.79}\Biggl)+P\Biggl(z>\dfrac{52-50}{0.79}\Biggl)\\
&=P(z<-2.53)+P(z>2.53)\\
&=0.0057+0.0057=0.0114
\end{align*}
\end{block}
\pause
\begin{block}{}
Poderíamos também reduzir $\alpha,$ \textbf{aumentando o tamanho da amostra.} Se $n = 16,$ então $\dfrac{\sigma}{\sqrt{n}}=\dfrac{2.5}{\sqrt{16}}=0.625.$ Logo,
\begin{align*}
z_{1}=\dfrac{48.5-50}{0.625}=-2.40\quad \textrm{e}\quad z_{2}=\dfrac{51.5-50}{0.625}=2.40
\end{align*}
e, $\alpha=P(z<-2.40)+P(z>2.40)=0.0082+0.0082=0.0164.$
\end{block}
\end{frame}

\begin{frame}{}
\frametitle{}
\begin{block}{}
\justifying
No entanto, na avaliação de um procedimento de teste de hipóteses, também é importante examinar a probabilidade do \textbf{erro tipo II}, que é denotado por $\beta.$ 
Lembremos que, $$ \beta=P(\textrm{Erro tipo II})=P(\textrm{Não rejeitar}\ \Ho\ \textrm{dado que}\ \Ho\ \textrm{é falsa})$$
\end{block}
\end{frame}

\begin{frame}{}
\frametitle{}
\begin{block}{}
\justifying
Para calcular $\beta$ (algumas vezes chamado de erro $\beta$), temos de ter uma hipótese alternativa específica fixada; ou seja, temos de ter um valor particular de 
$\mu$. Por exemplo, suponha que seja importante rejeitar a hipótese nula $\Ho: \mu = 50$ toda vez que a taxa média de queima $\mu$ seja maior do que 52 cm/s ou 
menor do que 48 cm/s.
\end{block}
\end{frame}

\begin{frame}{}
\frametitle{}
\begin{block}{}
\justifying
Poderíamos calcular a probabilidade de um erro tipo II, $\beta$, para os valores $\mu = 52$ e $\mu = 48$ e usar esse resultado para nos dizer 
alguma coisa acerca de como seria o desempenho do procedimento de teste. Por causa da simetria, só é necessário avaliar um dos dois casos. Isto é, encontrar a 
probabilidade de não rejeitar a hipótese nula $\Ho: \mu = 50$ cm/s, quando a média verdadeira, por exemplo, for $\mu = 52$ cm/s.
\end{block}
\end{frame}

\begin{frame}{}
\frametitle{}
\begin{block}{}
\justifying
A próxima Figura nos ajudará a calcular a probabilidade do erro tipo II, $\beta.$
\end{block}
\begin{center}
\begin{tikzpicture}
\begin{axis}[
  xmin=42,xmax=60,
  ymin=0,ymax=0.21,
  no markers, domain=42:60, samples=100,
  axis lines=left, xlabel=$\bar{x}$, ylabel=Densidade de Probabilidade,
  %every axis y label/.style={at=(current axis.above origin),anchor=south},
  %every axis x label/.style={at=(current axis.right of origin),anchor=west},
  height=8cm, width=14cm,
  ticks=both,
  xtick={44,46,48,50, 52,54,56,58}, ytick=\empty,
  enlargelimits=false, axis on top, %clip=false, 
  %grid = major
  ]
  \addplot [fill=cyan!20, draw=none, domain=48.5:51.5] {gauss(52,2)} \closedcycle;
  \addplot [very thick,cyan!50!black] {gauss(50,2)};
  \node [right] at (axis cs: 43, 0.15) {Sujeita a };
  \node [right] at (axis cs: 43, 0.13) {$H_{0}:\mu=50$ };
  \draw[very thick] (axis cs:48.5,0) -- (axis cs:48.5,0.15);
  \addplot [dashed,cyan!50!black] {gauss(52,2)};
  \node [right] at (axis cs: 55, 0.15) {Sujeita a };
  \node [right] at (axis cs: 55, 0.13) {$H_{1}:\mu=52$ };
  \draw[dashed] (axis cs:51.5,0) -- (axis cs:51.5,0.19);
\end{axis}
\end{tikzpicture}
\end{center}
\end{frame}

\begin{frame}{}
\frametitle{}
\begin{block}{}
\justifying
Um erro tipo II será cometido, se a média amostral $\bx$ cair entre 48,5 e 51,5, quando $\mu = 52.$ Como visto na Figura anterior, essa é apenas a probabilidade de 
$48,5 \leq \bar{X} \leq 51,5,$ quando a média verdadeira for $\mu = 52,$ ou a área sombreada sob a distribuição normal centralizada em $\mu = 52.$ Consequentemente, 
referindo-se à anterior, encontramos que $$\beta=P(48.5\leq\bar{X}\leq 51.5,\ \textrm{quando}\ \mu=52)$$
\end{block}
\end{frame}

\begin{frame}{}
\frametitle{}
\begin{block}{}
Os valores $z,$ correspondentes a $48,5$ e $51,5$, quando $\mu = 52,$ são
\begin{align*}
z_{1}=\dfrac{48.5-52}{0.79}=-4.43\quad \textrm{e}\quad z_{2}=\dfrac{51.5-52}{0.79}=-0.63
\end{align*}
logo, $$\beta=P(-4.43\leq z\leq -0.63)=0.2643.$$
\end{block}
\end{frame}

\begin{frame}{}
\frametitle{}
\begin{block}{}
\justifying
Assim, se estivermos testando $\Ho: \mu = 50$ contra $\Hi: \mu \neq 50,$ com $n = 10$ e o valor verdadeiro da média for $\mu = 52,$ a probabilidade de falharmos em 
rejeitar a falsa hipótese nula é $0,2643.$ Por simetria, se o valor verdadeiro da média for $\mu = 48,$ o valor de $\beta$ será também $0,2643.$ A probabilidade de 
cometer o erro tipo II, $\beta,$ aumenta rapidamente à medida que o valor verdadeiro de $\mu$ se aproxima do valor da hipótese feita. Por exemplo, veja a próxima 
Figura, em que o valor verdadeiro da média é $\mu = 50,5$ e o valor da hipótese é $\Ho: \mu = 50.$ O valor verdadeiro de $\mu$ está muito perto de 50 e o valor para 
$\beta$ é $$\beta=P(48.5\leq\bar{X}\leq 51.5,\ \textrm{quando}\ \mu=50.5)$$
\end{block}
\end{frame}

\begin{frame}{}
\frametitle{}
\begin{block}{}
\justifying
Os valores $z,$ correspondentes a $48,5$ e $51,5$, quando $\mu = 50.5,$ são
\begin{align*}
z_{1}=\dfrac{48.5-50.5}{0.79}=-2.53\quad \textrm{e}\quad z_{2}=\dfrac{51.5-50.5}{0.79}=1.27
\end{align*}
logo, $$\beta=P(-2.53\leq z\leq 1.27)=0.8923.$$
\end{block}
\end{frame}

\begin{frame}{}
\begin{center}
\begin{tikzpicture}
\begin{axis}[
  xmin=42,xmax=60,
  ymin=0,ymax=0.21,
  no markers, domain=42:60, samples=100,
  axis lines=left, xlabel=$\bar{x}$, ylabel=Densidade de Probabilidade,
  %every axis y label/.style={at=(current axis.above origin),anchor=south},
  %every axis x label/.style={at=(current axis.right of origin),anchor=west},
  height=8cm, width=14cm,
  ticks=both,
  xtick={44,46,48,50, 52,54,56,58}, ytick=\empty,
  enlargelimits=false, axis on top, %clip=false, 
  %grid = major
  ]
  \addplot [fill=cyan!20, draw=none, domain=48.5:51.5] {gauss(50.5,2)} \closedcycle;
  \addplot [very thick,cyan!50!black] {gauss(50,2)};
  \node [right] at (axis cs: 43, 0.15) {Sujeita a };
  \node [right] at (axis cs: 43, 0.13) {$H_{0}:\mu=50$ };
  \draw[dashed] (axis cs:48.5,0) -- (axis cs:48.5,0.125);
  \addplot [dashed,cyan!50!black] {gauss(50.5,2)};
  \node [right] at (axis cs: 55, 0.15) {Sujeita a };
  \node [right] at (axis cs: 55, 0.13) {$H_{1}:\mu=50.5$ };
  \draw[dashed] (axis cs:51.5,0) -- (axis cs:51.5,0.18);
\end{axis}
\end{tikzpicture}
\end{center}
\end{frame}

\begin{frame}{}
\frametitle{}
\begin{block}{}
\justifying
Assim, a probabilidade do erro tipo II é muito maior para o caso em que a média verdadeira é 50,5 centímetros por segundo do que para o caso em que a média é 
52 cm/s. Naturalmente, em muitas situações práticas, não estaríamos preocupados em cometer o erro tipo II se a média fosse “próxima” do valor utilizado na hipótese. 
Estaríamos muito mais interessados em detectar grandes diferenças entre a média verdadeira e o valor especificado na hipótese nula.
\end{block}
\end{frame}

\begin{frame}{}
\frametitle{}
\begin{block}{}
\justifying
A probabilidade do erro tipo II depende também do tamanho da amostra, $n.$ Suponha que a hipótese nula seja $\Ho: \mu = 50$ centímetros por segundo e que o valor 
verdadeiro da média seja $\mu = 52.$ Se o tamanho da amostra for aumentado de $n = 10$ para $n = 16,$ temos que o desvio-padrão de $\bar{X}$ é 
$\dfrac{\sigma}{\sqrt{n}}=\dfrac{2.5}{\sqrt{16}}=0.625.$ Logo,
\begin{align*}
z_{1}=\dfrac{48.5-52}{0.625}=-5.60\quad \textrm{e}\quad z_{2}=\dfrac{51.5-52}{0.625}=-0.80
\end{align*}
e, $\beta=P(-5.60\leq z\leq-0.80)=0.2119.$
\end{block}
\end{frame}

\begin{frame}{}
\frametitle{}
\begin{block}{}
\justifying
Lembre-se de que quando $n = 10$ e $\mu = 52,$ encontramos que $\beta = 0,2643;$ consequentemente, o aumento do tamanho da amostra resulta em uma diminuição 
na probabilidade de erro tipo II. Os resultados vistos até agora e outros poucos cálculos similares estão sumarizados na tabela abaixo. Os valores críticos são ajustados 
para manter $\alpha's$ iguais para n = 10 e n = 16. Esse tipo de cálculo é discutido mais adiante nas aulas.
\end{block}
\begin{table}[]
\begin{tabular}{|c|c|c|c|c|}
\hline
 RNR$\Ho$                                & n    & $\alpha$ & $\beta$ em $\mu=52$ & $\beta$ em $\mu=50.5$ \\ \hline
 $48.5\leq \bar{X}\leq 51.5$     & 10 & 0.0576 & 0.2643 & 0.8923 \\ \hline
 $48\leq \bar{X}\leq 52$           & 10 & 0.0114 & 0.5000 & 0.9705 \\ \hline
 $48.81\leq \bar{X}\leq 51.19$& 16 & 0.0576 & 0.0966 & 0.8606 \\ \hline
 $48.42\leq \bar{X}\leq 51.58$& 16 & 0.0114 & 0.2515 & 0.9578 \\ \hline
\end{tabular}
\end{table}

\end{frame}

\begin{frame}{}
\frametitle{}
\begin{block}{}
\justifying
A tabela anterior e a discussão anterior revelam quatro pontos importantes:
\begin{enumerate}
\item O tamanho da região crítica, e consequentemente a probabilidade do erro tipo I, $\alpha,$ pode sempre ser reduzido por meio da seleção apropriada dos valores 
críticos. \pause
\item Os erros tipo I e tipo II estão relacionados. Uma diminuição na probabilidade de um tipo de erro sempre resulta em um aumento da probabilidade do outro, desde 
que o tamanho da amostra, n, não varie. \pause
\item Um aumento no tamanho da amostra reduzirá $\beta,$ desde que $\alpha$ seja mantido constante. \pause
\item Quando a hipótese nula é falsa, $\beta$ aumenta à medida que o valor verdadeiro do parâmetro se aproxima do valor usado na hipótese nula. O valor de $\beta$ 
diminui à medida que aumenta a diferença entre a média verdadeira e o valor utilizado na hipótese.
\end{enumerate}
\end{block}
\end{frame}

\begin{frame}{}
\frametitle{}
\begin{block}{}
\justifying
Geralmente, o(a) analista controla a probabilidade $\alpha$ do erro tipo I quando ele ou ela seleciona os valores críticos. Assim, geralmente é fácil para o analista 
estabelecer a probabilidade de erro tipo I em (ou perto de) qualquer valor desejado. 
\end{block}
\pause
\begin{block}{}
\justifying
Uma vez que o analista pode controlar diretamente a probabilidade de rejeitar erroneamente $\Ho,$ sempre pensamos na rejeição da hipótese nula $\Ho$ como uma 
\textbf{conclusão forte.}
\end{block}
\end{frame}

\begin{frame}{}
\frametitle{}
\begin{block}{}
\justifying
Uma vez que podemos controlar a probabilidade de cometer um erro tipo I (ou nível de significância), uma questão lógica é que valor deve ser usado?
\end{block}
\pause
\begin{block}{}
\justifying
A probabilidade do \textbf{erro tipo I} é uma medida de risco, especificamente o risco de concluir que a hipótese nula é falsa quando ela realmente não é. Assim, o valor 
de $\alpha$ deve ser escolhido para refletir as consequências (econômicas, sociais etc.) de rejeitar incorretamente a hipótese nula. Frequentemente, isso é difícil de fazer, 
e o que tem evoluído muito na prática científica e de engenharia é usar o valor $\alpha = 0,05$ na maioria das situações, a menos que haja alguma informação disponível 
que indique que essa é uma escolha não apropriada. 
\end{block}
\end{frame}

\begin{frame}{}
\frametitle{}
\begin{block}{}
\justifying
Por outro lado, a probabilidade $\beta$ do erro tipo II não é constante, mas depende do valor verdadeiro do parâmetro. Ela depende também do tamanho da amostra 
que tenhamos selecionado. Pelo fato de a probabilidade $\beta$ do erro tipo II ser uma função do tamanho da amostra e da extensão com que a hipótese nula $\Ho$ 
seja falsa, costuma-se pensar na aceitação de $\Ho$ como uma conclusão fraca, a menos que saibamos que $\beta$ seja aceitavelmente pequena. 
\end{block}
\end{frame}

\begin{frame}{}
\frametitle{}
\begin{block}{}
\justifying
Consequentemente, em vez de dizer ``aceitar $\Ho$'', preferimos a terminologia ``\textbf{Não rejeitar }$\Ho$''. Falhar em rejeitar $\Ho$ implica que não encontramos evidência 
suficiente para rejeitar $\Ho,$ ou seja, para fazer uma afirmação forte. Falhar em rejeitar $\Ho$ não significa necessariamente que haja uma alta probabilidade de que 
$\Ho$ seja verdadeira. Isso pode significar simplesmente que mais dados são requeridos para atingir uma conclusão forte, o que pode ter implicações importantes para 
a formulação das hipóteses.
\end{block}
\end{frame}

\begin{frame}{}
\frametitle{}
\begin{block}{}
\justifying
Existe uma analogia útil entre teste de hipóteses e um julgamento por jurados. Em um julgamento, o réu é considerado inocente (isso é como considerar a hipótese 
nula verdadeira). Se forte evidência for encontrada do contrário, o réu é declarado culpado (rejeitamos a hipótese nula). Se não houver \textbf{suficiente} evidência, o réu é 
declarado não culpado. Isso não é o mesmo de provar a inocência do réu; assim, tal qual falhar em rejeitar a hipótese nula, essa é uma conclusão fraca.
\end{block}
\end{frame}

\begin{frame}{}
\frametitle{}
\begin{block}{}
\justifying
Um importante conceito de que faremos uso é a \textbf{potência} ou \textbf{poder} de um teste estatístico:
\end{block}
\begin{tikzpicture}
\node[draw,align=center, fill=gray!30] at (0,-1) {A potência ou poder de um teste estatístico é a probabilidade de rejeitar\\ a hipótese nula H0, quando a hipótese alternativa 
for verdadeira.};
\end{tikzpicture}
\end{frame}

\begin{frame}{}
\frametitle{}
\begin{block}{}
\justifying
A potência é calculada como $1-\beta,$ e pode ser interpretada como a probabilidade de rejeitar corretamente uma hipótese nula falsa. Frequentemente, 
comparamos testes estatísticos por meio da comparação de suas propriedades de potência. Por exemplo, considere o problema da taxa de queima de propelente, quando 
estamos testando $\Ho: \mu = 50$ cm/s contra $\Hi: \mu \neq 50$ cm/s. Suponha que o valor verdadeiro da média seja $\mu = 52.$ Quando $n = 10,$ encontramos 
que $\beta = 0.2643;$ logo, a potência desse teste é $1-\beta = 1 - 0.2643 = 0.7357,$ quando $\mu = 52.$
\end{block}
\end{frame}

\begin{frame}{}
\frametitle{}
\begin{block}{}
\justifying
A potência é uma medida muito descritiva e concisa da sensibilidade de um teste estatístico, em que por sensibilidade entendemos a habilidade do teste de detectar 
diferenças. Nesse caso, a sensibilidade do teste para detectar a diferença entre a taxa média de queima de 50 centímetros por segundo e 52 centímetros por segundo é 
$0.7357.$ Isto é, se a média verdadeira for realmente 52 centímetros por segundo, esse teste rejeitará corretamente $\Ho: \mu = 50$ e ``detectará'' essa diferença em 
$73,57\%$ das vezes. Se esse valor de potência for julgado como muito baixo, o analista poderá aumentar tanto $\alpha$ como o tamanho da amostra $n.$

\end{block}
\end{frame}

\begin{frame}{}
\frametitle{Referências Bibliográficas}
\bibliography{bibliografia}
\end{frame}

\end{document}


