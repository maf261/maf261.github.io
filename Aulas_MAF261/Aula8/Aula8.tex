\documentclass[14pt,aspectratio=1610]{beamer}

\usepackage[brazil]{babel}
\usepackage[utf8]{inputenc}
%\UseRawInputEncoding
\usepackage[T1]{fontenc}
\usepackage{Sweave}
\usepackage{animate}
\usepackage{amsbsy}
\usepackage{amsfonts}
\usepackage{amsmath}
\usepackage{amssymb}
\usepackage{amsthm}
\usepackage[toc,page,title,titletoc]{appendix}
\usepackage[fixlanguage]{babelbib}
%\usepackage[pdftex]{color}
\usepackage{dsfont}
\usepackage{esvect}
\usepackage[labelfont=bf]{caption}
\usepackage{float}
\usepackage[Glenn]{fncychap}%Sonny %Conny %Lenny %Glenn %Renje %Bjarne %Bjornstrup
%\usepackage{geometry, calc, color, setspace}%
%\geometry{a4paper, headsep=1.0cm, footskip=1cm, lmargin=3cm, rmargin=2cm, tmargin=3cm, bmargin=2cm}
\usepackage{graphicx}
\usepackage{indentfirst}%Para indentar os parágrafos automáticamente
\usepackage{lipsum}
\usepackage{longtable}
\usepackage{mathtools}
\usepackage{listings}%Inserir codigo do R no latex
\usepackage{multirow}
\usepackage{multicol}
\usepackage{natbib}
\bibliographystyle{abbrvnat3}
\usepackage[figuresright]{rotating}
\usepackage{spalign}
%\usepackage{pgfpages}
\usepackage{pgfplots}
\usepackage{tikz}
\usepackage{color, colortbl}
\usepackage{ragged2e}%para justificar o texto dentro de algum ambiente
\definecolor{Gray}{gray}{0.9}
\definecolor{LightCyan}{rgb}{0.88,1,1}


\usepackage[all]{xy}
\usepackage{hyperref,bookmark}
\hypersetup{
  colorlinks=true,
  linkcolor=blue,
  citecolor=red,
  filecolor=blue,
  urlcolor=blue,
}

\usetheme{Boadilla}
%\usecolortheme[RGB={193,0,0}]{structure}

%\setbeamertemplate{footline}[frame number]
%\setbeamertemplate{footline}[text line]{%
%  \parbox{\linewidth}{\vspace*{-8pt}\hfill\date{}\hfill\insertshortauthor\hfill\insertpagenumber}}
\beamertemplatenavigationsymbolsempty
\renewcommand{\vec}[1]{\mbox{\boldmath$#1$}}
\newtheorem{Teorema}{Teorema}
\newtheorem{Proposicao}{Proposição}
\newtheorem{Definicao}{Definição}
\newtheorem{Corolario}{Corolário}
\newtheorem{Demonstracao}{Demonstração}
\newcommand{\bx}{\ensuremath{\bar{x}}}
\newcommand{\Ho}{\ensuremath{H_{0}}}
\newcommand{\Hi}{\ensuremath{H_{1}}}


\apptocmd{\frame}{}{\justifying}{} % Allow optional arguments after frame.

\title{MAF 261 - Estatística Experimental}
\author{Prof. Fernando de Souza Bastos}
\institute{Instituto de Ciências Exatas e Tecnológicas\texorpdfstring{\\ Universidade Federal de Viçosa}{}\texorpdfstring{\\ Campus UFV - Florestal}{}}
\date{2018}
\newcommand\mytext{Aula 8}
\newcommand\mytextt{Fernando de Souza Bastos}
\makeatletter
\setbeamertemplate{footline}
{
  \leavevmode%
  \hbox{%
  \begin{beamercolorbox}[wd=.333333\paperwidth,ht=2.25ex,dp=1ex,center]{author in head/foot}%
    \usebeamerfont{author in head/foot}\mytext
  \end{beamercolorbox}%
  \begin{beamercolorbox}[wd=.333333\paperwidth,ht=2.25ex,dp=1ex,center]{title in head/foot}%
    \usebeamerfont{title in head/foot}\mytextt
  \end{beamercolorbox}%
  \begin{beamercolorbox}[wd=.333333\paperwidth,ht=2.25ex,dp=1ex,right]{date in head/foot}%
    \usebeamerfont{date in head/foot}\insertshortdate{}\hspace*{2em}
    \insertframenumber{} / \inserttotalframenumber\hspace*{2ex} 
  \end{beamercolorbox}}%
  \vskip0pt%
}
\makeatother


\providecommand{\arcsin}{} \renewcommand{\arcsin}{\hspace{2pt}\textrm{arcsen}}
\providecommand{\sin}{} \renewcommand{\sin}{\hspace{2pt}\textrm{sen}}
%\newtheorem{Teorema}{Teorema}
%\newtheorem{Proposicao}{Proposição}
%\newtheorem{Definicao}{Definição}
%\newtheorem{Corolario}{Corolário}
%\newtheorem{Demonstracao}{Demonstração}

% Layout da pagina
\hypersetup{pdfpagelayout=SinglePage}
\begin{document}
\Sconcordance{concordance:Aula8.tex:Aula8.Rnw:%
1 366 1}


\frame{\titlepage}

\begin{frame}{}
\frametitle{\bf Sumário}
\tableofcontents
\end{frame}

\section{Princípios Básicos da Experimentação e Contrastes}
\begin{frame}{}
\frametitle{}
\begin{block}{}
\justifying
O estudo de contrastes é muito importante na Estatística Experimental, principalmente quando o experimento em análise é composto por mais do que dois tratamentos. Com o uso de contrastes é possível ao pesquisador estabelecer comparações, entre tratamentos ou grupos de tratamentos, que sejam de interesse. Vamos estudar os fundamentos para estabelecer grupos de contrastes, obter a estimativa para cada contraste estabelecido, bem com estimar a variabilidade associada a cada um destes contrastes. 
\end{block}
\end{frame}

\begin{frame}{}
\frametitle{}
\begin{block}{}
\justifying
 {\bf Definição:} Uma função linear de médias populacionais de tratamentos dada por $$C=a_{1}\mu_{1}+a_{2}\mu_{2}+\cdots+a_{n}\mu_{n}$$ é dita ser um contraste se satisfizer:
$${\displaystyle \sum_{i=1}^{n}a_{i}=0}$$ 
\end{block}
\end{frame}

\section{Estimador do Contraste}
\begin{frame}{}
\frametitle{}
\begin{block}{}
\justifying
Na prática, geralmente não se conhece os valores das médias populacionais $\mu_{i}$,
mas suas estimativas. Daí, em Estatística Experimental, não trabalhamos com o contraste
$C$ mas com o seu estimador $\hat{C}$, que também é uma função linear de médias obtidas por meio de experimentos ou amostras. Assim tem-se que o estimador para o contraste de
médias é dado por: 
$$\hat{C}=a_{1}\hat{\mu}_{1}+a_{2}\hat{\mu}_{2}+\cdots+a_{n}\hat{\mu}_{n}$$
\end{block}
\end{frame}

\begin{frame}{}
\frametitle{Exemplo}
\begin{block}{}
\justifying
 É contraste?
 \begin{enumerate}
 \item $C_{1}=\mu_{1}-\mu_{2}$ \pause
 \item $C_{2}=\mu_{1}+2\mu_{2}+5\mu_{3}-4\mu_{4}-3\mu_{5}$ \pause
 \item $C_{1}=\mu_{1}+\mu_{2}-2\mu_{3}$
 \end{enumerate}
\end{block}
\end{frame}

\begin{frame}{}
\frametitle{}
\begin{block}{}
Num experimento de consórcio na cultura do abacaxi, com 5 repetições, as médias de
produção de frutos de abacaxi (em t/ha), foram as seguintes:
\begin{table}[]
\begin{tabular}{lc}
\hline
Tratamentos&$\hat{\mu}_{i}$\\
\hline
$1 - $Abacaxi $(0,90\ x 0,30m)$ monocultivo& 53,5\\
$2 - $Abacaxi $(0,80 x 0,30m)$ monocultivo& 56,5\\
$3 - $Abacaxi $(0,80 x 0,30m) +$ amendoim & 62,0\\
$4 - $Abacaxi $(0,80 x 0,30m) +$ feijão   & 60,4\\
\hline
\end{tabular}
\end{table}
Pede-se obter as estimativas dos seguintes contrastes:
$$C_{1}=\mu_{1}+\mu_{2}-\mu_{3}-\mu_{4};\ C_{2}=\mu_{1}-\mu_{2};\ C_{3}=\mu_{3}-\mu_{4}$$
\end{block}
\end{frame}

\section{Variância do Estimador de um Contraste}
\begin{frame}{}
\frametitle{}
\begin{block}{}
Considere o estimador do contraste C, dado por:
$$\hat{C}=a_{1}\hat{\mu}_{1}+a_{2}\hat{\mu}_{2}+\cdots+a_{n}\hat{\mu}_{n}$$
A variância do estimador do contraste é dada por:
$$V(\hat{C})=V(a_{1}\hat{\mu}_{1}+a_{2}\hat{\mu}_{2}+\cdots+a_{n}\hat{\mu}_{n})$$
Admitindo independência entre as médias, temos:
\begin{align*}
V(\hat{C})=V(a_{1}\hat{\mu}_{1})+V(a_{2}\hat{\mu}_{2})+\cdots+V(a_{n}\hat{\mu}_{n})\\
\end{align*}
Logo,
\begin{align*}
V(\hat{C})=a_{1}^{2}V(\hat{\mu}_{1})+a_{2}^{2}V(\hat{\mu}_{2})+\cdots+a_{n}^{2}V(\hat{\mu}_{n})
\end{align*}
\end{block}
\end{frame}

\begin{frame}{}
\frametitle{}
\begin{block}{}
\justifying
Lembrando que $\hat{\mu}_{i}={\displaystyle \dfrac{\sum_{j=1}^{r_{i}}x_{j}}{r_{i}}}$ temos que $V(\hat{\mu}_{i})=\dfrac{\sigma_{i}^{2}}{r_{i}},$ assim:
\begin{align*}
V(\hat{C})=a_{1}^{2}\dfrac{\sigma_{1}^{2}}{r_{1}}+a_{2}^{2}\dfrac{\sigma_{2}^{2}}{r_{2}}+\cdots+a_{n}^{2}\dfrac{\sigma_{n}^{2}}{r_{n}}
\end{align*}
Admitindo-se homogeneidade de variâncias, ou seja, 
$\sigma_{1}^{2}=\sigma_{2}^{2}=\cdots=\sigma_{n}^{2}=\sigma^{2}$, então
\begin{align*}
V(\hat{C})=\Biggl(\dfrac{a_{1}^{2}}{r_{1}}+\dfrac{a_{2}^{2}}{r_{2}}+\cdots+\dfrac{a_{n}^{2}}{r_{n}}\Biggl)\sigma^{2}=\sigma^{2}\sum_{i=1}^{n}\dfrac{a_{i}^{2}}{r_{i}}
\end{align*}
\end{block}
\end{frame}

\begin{frame}{}
\frametitle{}
\begin{block}{}
\justifying
Na prática, geralmente, não se conhece a variância $\sigma^{2}$, mas sua estimativa a qual é obtida por meio de dados experimentais. Esta estimativa é denominada como estimador comum $S_{c}^{2}.$ Então, o que normalmente se obtém é o valor do estimador da variância do estimador do contraste, que é obtida por:
$${\displaystyle \hat{V}(\hat{C})=S_{c}^{2}\sum_{i=1}^{n}\dfrac{a_{i}^{2}}{r_{i}}}$$
\end{block}
\end{frame}

\section{Contrastes Ortogonais}
\begin{frame}{}
\frametitle{}
\begin{block}{}
\justifying
Em algumas situações desejamos testar um grupo de contrastes relacionados com o experimento em estudo. Alguns tipos de testes indicados para este objetivo, necessitam
que os contrastes, que compõem o grupo a ser testado, sejam ortogonais entre si. A ortogonalidade entre os contrastes indica independência linear na comparação estabelecida por um contraste com a comparação estabelecida pelos outros contrastes.
\end{block}
\end{frame}

\begin{frame}{}
\frametitle{}
\begin{block}{}
\justifying
Sejam os estimadores dos contrastes de $C_{1}$ e $C_{2}$ dados, respectivamente, por:
\begin{align*}
\hat{C}_{1}&=a_{1}\hat{\mu}_{1}+a_{2}\hat{\mu}_{2}+\cdots+a_{n}\hat{\mu}_{n}\\
\hat{C}_{2}&=b_{1}\hat{\mu}_{1}+b_{2}\hat{\mu}_{2}+\cdots+b_{n}\hat{\mu}_{n}
\end{align*}
A covariância entre $\hat{C}_{1}$ e $\hat{C}_{2},$ supondo independência entre tratamentos, é obtida por 
$Cov(\hat{C}_{1},\hat{C}_{2})=a_{1}b_{1}V(\hat{\mu}_{1})+a_{2}b_{2}V(\hat{\mu}_{2})+
\cdots+a_{n}b_{n}V(\hat{\mu}_{n}).$ E como já vimos, a variância da média amostral é dada por $V(\hat{\mu}_{i})=\frac{\sigma_{i}^{2}}{r_{i}},$ para $i=1,2,\cdots,n.$ Logo,
$$Cov(\hat{C}_{1},\hat{C}_{2})=a_{1}b_{1}\dfrac{\sigma_{1}^{2}}{r_{1}}+a_{2}b_{2}\dfrac{\sigma_{2}^{2}}{r_{2}}+
\cdots+a_{n}b_{n}\dfrac{\sigma_{n}^{2}}{r_{n}}.$$
\end{block}
\end{frame}

\begin{frame}{}
\frametitle{}
\begin{block}{}
\justifying
Admitindo que exista homogeneidade de variâncias entre os tratamentos, ou seja, 
$\sigma_{1}^{2}=\sigma_{2}^{2}=\cdots=\sigma_{n}^{2}=\sigma^{2}$, então,
$$Cov(\hat{C}_{1},\hat{C}_{2})=\Biggl(\dfrac{a_{1}b_{1}}{r_{1}}+\dfrac{a_{2}b_{2}}{r_{2}}+\cdots+\dfrac{a_{n}b_{n}}{r_{n}}\Biggl)\sigma^{2}=\sigma^{2}{\displaystyle \sum_{i=1}^{n}\dfrac{a_{i}b_{i}}{r_{i}}}$$
\end{block}
\end{frame}

\begin{frame}{}
\frametitle{}
\begin{block}{}
\justifying
Sabe-se ainda que, se duas variáveis aleatórias são independentes, a covariância entre
elas é igual a zero. Assim, se $\hat{C}_{1}$ e $\hat{C}_{2}$ são independentes, a covariância entre eles é igual a zero, isto é: $$Cov(\hat{C}_{1},\hat{C}_{2})=0$$
Para que a covariância seja nula, é necessário, portanto que:
$${\displaystyle \sum_{i=1}^{n}\dfrac{a_{i}b_{i}}{r_{i}}}=0$$
\end{block}
\end{frame}

\begin{frame}{}
\frametitle{}
\begin{block}{}
\justifying
Esta é a condição de ortogonalidade entre dois contrastes para um experimento
com número diferente de repetições para os tratamentos. Para um experimento com o
mesmo número de repetições, satisfazendo as mesmas pressuposições (médias
independentes e homogeneidade de variâncias), a condição de ortogonalidade se resume
a: $${\displaystyle \sum_{i=1}^{n}a_{i}b_{i}}=0$$
\end{block}
\end{frame}

\begin{frame}{}
\frametitle{}
\begin{block}{}
\justifying
Para um experimento com $I$ tratamentos, podem ser formados vários grupos de
contrastes ortogonais, no entanto cada grupo deverá conter no máximo $(I-1)$ contrastes
ortogonais, o que corresponde ao número de graus de liberdade para tratamentos.
Dentro de um grupo de contrastes ortogonais, todos os contrastes tomados dois a
dois, serão também ortogonais.
\end{block}
\end{frame}

\section{Métodos para Obtenção de Grupos de Contrastes Mutuamente Ortogonais}
\begin{frame}{}
\frametitle{Obtenção por Meio de Sistema de Equações Lineares}
\begin{block}{}
\justifying
Neste método, deve-se estabelecer, a princípio, um contraste que seja de interesse
e, a partir deste é que os demais são obtidos. Por meio da imposição da condição de
ortogonalidade e da condição para ser um contraste, obtém-se equações lineares, cujas
incógnitas são os coeficientes das médias que compõem o contraste. Como o número de
incógnitas é superior ao número de equações existentes, será sempre necessário atribuir
valores a algumas incógnitas. É desejável que os valores a serem atribuídos, permitam
que os coeficientes sejam números inteiros.
\end{block}
\end{frame}

\begin{frame}{}
\frametitle{}
\begin{block}{}
\justifying
Por meio desta metodologia, é possível estabelecer facilmente um grupo de
contrastes ortogonais. A metodologia pode ser resumida nos seguintes passos
(BANZATTO e KRONKA, 1989):
\begin{enumerate}
\item Divide-se o conjunto das médias de todos os tratamentos do experimento em dois
grupos. O primeiro contraste é obtido pela comparação das médias de um grupo contra
as médias do outro grupo. Para isso atribui-se sinais positivos para membros de um grupo e negativos para membros do outro grupo.\pause
\item Dentro de cada grupo formado no passo anterior, que possui mais que uma média,
aplica-se o passo 1, subdividindo-os em subgrupos. Repete-se este passo até que se
forme subgrupos com apenas uma média. Ao final, deveremos ter formado $(I-1)$
comparações.
\end{enumerate}
\end{block}
\end{frame}

\begin{frame}{}
\frametitle{}
\begin{block}{}
\justifying
Para se obter os coeficientes que multiplicam cada média que compõem os contrastes estabelecidos, deve-se, para cada contraste:
\begin{enumerate}
\item Verificar o número de parcelas experimentais envolvidas no $1^{\circ}$ grupo, digamos $g_{1},$ e o número de parcelas experimentais envolvidas no $2^{\circ}$ grupo, digamos $g_2.$ Calcula-se o mínimo múltiplo comum (m.m.c.) entre $g_1$ e $g_2.$ \pause
\item Dividir o m.m.c. por $g_1.$ O resultado será o coeficiente de cada média do $1^{\circ}$ grupo. \pause
\item Dividir o m.m.c. por $g_2.$ O resultado será o coeficiente de cada média do $2^{\circ}$ grupo. \pause
\item Multiplicar os coeficientes obtidos pelo número de repetições da respectiva média. Se possível, simplificar os coeficientes obtidos por uma constante. No caso em que o
número de repetições é igual para todos os tratamentos, este passo pode ser eliminado.
\end{enumerate}
\end{block}
\end{frame}

\begin{frame}{}
\frametitle{Exemplo}
\begin{block}{}
\justifying
\begin{align*}
C_{1}&=(T_{1},T_{2},T_{3},T_{4})\ vs\ T_{5}\\
C_{2}&=(T_{1},T_{2},T_{3})\ vs\ T_{4}\\
C_{3}&=(T_{1},T_{2})\ vs\ T_{3}\\
C_{4}&=(T_{1})\ vs\ T_{2}\\
\end{align*}
Em que $T_{1}$ possui 6 repetições, $T_{2}$ possui 6, $T_{3}$ possui 4, $T_{4}$ possui 5 e $T_{6}$ possui 6.
\end{block}
\end{frame}

\end{document}
