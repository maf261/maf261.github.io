\documentclass[14pt,aspectratio=1610]{beamer}

\usepackage[brazil]{babel}
\usepackage[utf8]{inputenc}
\usepackage[T1]{fontenc}
\mathchardef\hyphenmathcode=\mathcode`\-
\usepackage{listings}
%\usepackage{xr-hyper}

\usepackage{sansmathaccent}
\pdfmapfile{+sansmathaccent.map}
\usepackage{amsbsy}
\usepackage{amsfonts}
\usepackage{amsmath}
\usepackage{amssymb}
\usepackage{amsthm}
\usepackage[toc,page,title,titletoc]{appendix}
\usepackage[fixlanguage]{babelbib}
%\usepackage[pdftex]{color}
\usepackage{dsfont}
\usepackage{esvect}
\usepackage[labelfont=bf]{caption}
\usepackage{float}
\usepackage[Glenn]{fncychap}%Sonny %Conny %Lenny %Glenn %Renje %Bjarne %Bjornstrup
%\usepackage{geometry, calc, color, setspace}%
%\geometry{a4paper, headsep=1.0cm, footskip=1cm, lmargin=3cm, rmargin=2cm, tmargin=3cm, bmargin=2cm}
\usepackage{graphicx}
\usepackage{indentfirst}%Para indentar os parágrafos automáticamente
\usepackage{lipsum}
\usepackage{longtable}
\usepackage{mathtools}
\usepackage{multirow}
\usepackage{multicol}
\usepackage{natbib}
\bibliographystyle{abbrvnat3}
\usepackage[figuresright]{rotating}
\usepackage{spalign}
%\usepackage{pgfpages}
\usepackage{pgfplots}
\usepackage{tikz}
\usepackage{color, colortbl}
\usepackage{ragged2e}%para justificar o texto dentro de algum ambiente
\definecolor{Gray}{gray}{0.9}
\definecolor{LightCyan}{rgb}{0.88,1,1}


\usepackage[all]{xy}
\usepackage{hyperref,bookmark}
\hypersetup{
  colorlinks=true,
  linkcolor=blue,
  citecolor=red,
  filecolor=blue,
  urlcolor=blue,
}


%\setcitestyle{authoryear, open={(},close={)}}
%\usepackage{pgf}
%\usepackage[small,bf,singlelinecheck=off]{caption}
%\usepackage[figuresright]{rotating}

%\usepackage[font=Times,timeinterval=10, timeduration=2.0, timedeath=0, fillcolorwarningsecond=white!60!yellow,
%timewarningfirst=50,timewarningsecond=80,resetatpages=2]{tdclock}
\usetheme{Madrid}
\usecolortheme[RGB={193,0,0}]{structure}

%\setbeamertemplate{footline}[frame number]
%\setbeamertemplate{footline}[text line]{%
%  \parbox{\linewidth}{\vspace*{-8pt}\hfill\date{}\hfill\insertshortauthor\hfill\insertpagenumber}}
\beamertemplatenavigationsymbolsempty
\renewcommand{\vec}[1]{\mbox{\boldmath$#1$}}
\newtheorem{Teorema}{Teorema}
\newtheorem{Proposicao}{Proposição}
\newtheorem{Definicao}{Definição}
\newtheorem{Corolario}{Corolário}
\newtheorem{Demonstracao}{Demonstração}
\newcommand{\bx}{\ensuremath{\bar{x}}}
\newcommand{\Ho}{\ensuremath{H_{0}}}
\newcommand{\Hi}{\ensuremath{H_{1}}}


\apptocmd{\frame}{}{\justifying}{} % Allow optional arguments after frame.

\title{MAF 261 - Estatística Experimental}
\author{Prof. Fernando de Souza Bastos}
\institute{Instituto de Ciências Exatas e Tecnológicas\texorpdfstring{\\ Universidade Federal de Viçosa}{}\texorpdfstring{\\ Campus UFV - Florestal}{}}
\date{12/03/2019}
\newcommand\mytext{Aula 1}
\newcommand\mytextt{Fernando de Souza Bastos}
\makeatletter
\setbeamertemplate{footline}
{
  \leavevmode%
  \hbox{%
  \begin{beamercolorbox}[wd=.333333\paperwidth,ht=2.25ex,dp=1ex,center]{author in head/foot}%
    \usebeamerfont{author in head/foot}\mytext
  \end{beamercolorbox}%
  \begin{beamercolorbox}[wd=.333333\paperwidth,ht=2.25ex,dp=1ex,center]{title in head/foot}%
    \usebeamerfont{title in head/foot}\mytextt
  \end{beamercolorbox}%
  \begin{beamercolorbox}[wd=.333333\paperwidth,ht=2.25ex,dp=1ex,right]{date in head/foot}%
    \usebeamerfont{date in head/foot}\insertshortdate{}\hspace*{2em}
    \insertframenumber{} / \inserttotalframenumber\hspace*{2ex} 
  \end{beamercolorbox}}%
  \vskip0pt%
}
\makeatother


\providecommand{\arcsin}{} \renewcommand{\arcsin}{\hspace{2pt}\textrm{arcsen}}
\providecommand{\sin}{} \renewcommand{\sin}{\hspace{2pt}\textrm{sen}}
%\newtheorem{Teorema}{Teorema}
%\newtheorem{Proposicao}{Proposição}
%\newtheorem{Definicao}{Definição}
%\newtheorem{Corolario}{Corolário}
%\newtheorem{Demonstracao}{Demonstração}

% Layout da pagina
\hypersetup{pdfpagelayout=SinglePage}
\begin{document}




\frame{\titlepage}

\begin{frame}{}
\frametitle{\bf Sumário}
\tableofcontents
\end{frame}


\begin{frame}{}
\frametitle{}

\begin{block}{Filosofia do Curso:}
\begin{itemize}
    \item Tomar ciência da importância em aprender Estatística;\pause
    \item Importância de aprender a trabalhar com softwares para análises estatísticas;\pause
    \item Os exercícios devem ser replicados em softwares;\pause
    \item O estudo e a prática levam a perfeição!
\end{itemize}
\end{block}
\end{frame}
\section{Introdução}
\begin{frame}{}
\frametitle{Introdução}
\begin{block}{}
\justifying
Experimentação é uma área da estatística que estuda o planejamento, coleta de dados, execução, análise e interpretação de resultados provenientes de experimentos. 
Ela oferece suporte probabilístico ao pesquisador e lhe permite, com grau de incerteza conhecido, fazer inferências sobre o comportamento de diferentes fenômenos 
da natureza.
\end{block}
\end{frame}

\begin{frame}{}
\frametitle{Experimento:}
\begin{block}{}
\justifying
\begin{itemize}
\item É a ação ou o efeito de experimentar; \pause
\item É conhecimento adquirido pela prática da observação ou exercício; \pause
\item São ensaios, tentativas para verificar ou demonstrar qualquer coisa. Pode também ser definido como um teste ou um conjunto de testes realizados com a finalidade 
de verificar uma hipótese ou induzir uma hipótese a partir da observação de um fenômeno natural ou provocado. 
\end{itemize}
\end{block}
\end{frame}

\begin{frame}{}
\frametitle{Objetivos do Experimento:}
\begin{block}{}
Em geral, os objetivos de um experimento são:
\begin{itemize}
    \item Determinar quais \textbf{fatores} mais afetam a variável de interesse;\pause
    \item Determinar os valores necessários dos \textbf{fatores controláveis} de forma a obter a saída mais próxima do valor nominal desejado;\pause
    \item Determinar que valores atribuir aos fatores controláveis do processo, de forma a tornar pequena a \textbf{variabilidade} na saída;\pause
    \item Determinar que valores atribuir aos fatores controláveis do processo, de forma a torna-lo mais robusto aos efeitos das \textbf{variáveis não controláveis};\pause
%    \item Determinar os valores ótimos dos fatores controláveis do processo, para torna-lo mais econômico ou para melhorar as características tecnológicas do produto 
%		resultante;\pause
    \item Diminuição de \textbf{custos}, \textbf{melhoria de processos}, \textbf{aumento da produção}, \textbf{diminuição da variabilidade} e diversos outros.
\end{itemize}
\end{block}
\end{frame}

\begin{frame}{}
\frametitle{Diretrizes para o planejamento de um experimento}
\begin{block}{}
\justifying
O \textbf{planejamento} e a correta execução de um experimento são pontos essenciais para o seu sucesso. Sem planejamento, a avaliação dos resultados e a inferência 
podem ser prejudicadas ou \textbf{inviabilizadas}. 
\end{block}
\end{frame}

\begin{frame}{}
\frametitle{Diretrizes para o planejamento de um experimento}
\begin{block}{Reconhecimento, relato do problema, conjectura!}
\justifying
Torna-se bastante difícil reconhecer e aceitar a existência de um problema, se não ficar claro para todos qual é o problema e quais os objetivos a serem alcançados com 
a solução do mesmo. Neste ponto, é importante \textbf{conversar} com todos os envolvidos no processo e elaborar uma hipótese original que motivará o experimento.
\end{block}
\end{frame}

\begin{frame}{}
\frametitle{Diretrizes para o planejamento de um experimento}
\begin{block}{Escolha dos fatores, dos níveis, da variável resposta e o planejamento experimental.}
\justifying
É importante investigar todos os fatores que possam ser importantes no desenvolvimento do processo ou produto, a variável que queremos medir e o \textbf{delineamento} 
(metodologia) utilizada para investigar e tentar responder a conjectura.
\end{block}
\end{frame}

%\begin{frame}{}
%\frametitle{Diretrizes para o planejamento de um experimento}
%\begin{block}{Seleção da variável resposta}
%O que queremos medir?
%\end{block}
%\end{frame}

%\begin{frame}{}
%\frametitle{Diretrizes para o planejamento de um experimento}
%\begin{block}{Escolha do planejamento experimental}
%A escolha do planejamento envolve consideração pelo tamanho da amostra (número de %replicações), seleção de uma ordem adequada de rodadas para as tentativas experimentais, %ou se a formação de blocos ou outras restrições de aleatorização estão envolvidas.
%\end{block}
%\end{frame}

\begin{frame}{}
\frametitle{Diretrizes para o planejamento de um experimento}
\begin{block}{Experimento}
\justifying
Quando da realização do experimento, é de vital importância \textbf{monitorar} o processo, para garantir que tudo esteja sendo feito de acordo com o planejamento. Erros no 
procedimento experimental neste estágio, em geral, destruirão a validade do experimento.
\end{block}
\end{frame}

\begin{frame}{}
\frametitle{Diretrizes para o planejamento de um experimento}
\begin{block}{Análise dos dados}
\justifying
Métodos estatísticos devem ser usados para analisar os dados, de modo que os resultados e conclusões sejam objetivos. Se o experimento foi planejado corretamente 
e se foi realizado de acordo com o planejamento, então os tipos de métodos estatísticos exigidos não são
complicados.
\end{block}
\end{frame}

\begin{frame}{}
\frametitle{Diretrizes para o planejamento de um experimento}
\begin{block}{Conclusões e recomendações}
\justifying
Uma vez analisados os dados, o experimento deve acarretar conclusões práticas sobre os resultados e recomendar um curso de ação. \textbf{Métodos gráficos} são, em geral, 
usados neste estágio, particularmente na apresentação dos resultados para outras pessoas. Sequências de acompanhamento e testes de confirmação podem ser também 
realizados para validar as conclusões do experimento.
\end{block}
\end{frame}

%\begin{frame}{}
%\frametitle{Fases de um experimento}
%\begin{block}{}
%Podemos dividir um experimento em quatro fases:
%\begin{description}
% \item[P:] Planejamento do experimento, também conhecido como delineamento do %experimento;
% \item[D:] É a realização do experimento, de acordo com o que foi planejado;
% \item[C:] Análise dos resultados obtidos no experimento via estudos estatísticos e %análise de variância;
% \item[A:] Conclusões e recomendações.
%\end{description}
%\end{block}
%\end{frame}

\begin{frame}{}
\frametitle{}
\begin{block}{}
\justifying
 {\bf Todos os experimentos devem ser bem planejados.} Infelizmente, muitos não seguem este princípio e, como resultado, fontes valiosas de recursos, de tempo e 
de mão de obra são perdidos. Experimentos estatisticamente bem planejados permitem ganhos de eficiência e economia no processo experimental e o uso de métodos 
estatísticos na avaliação de resultados resulta na objetividade cientifica na obtenção de conclusões.
 \end{block}
\end{frame}

\section{Inferência Estatística}
\begin{frame}{}
\frametitle{Inferência Estatística}
\begin{block}{}
\justifying
A \textbf{inferência estatística} consiste em métodos usados para tomar decisões ou tirar 
conclusões acerca de uma população. Para tirar conclusões, esses métodos utilizam a 
informação contida em uma \textbf{amostra} proveniente da \textbf{população}. Dividimos a inferência em 
duas grande áreas: Estimação de parâmetros e teste de hipóteses. Vejamos exemplos 
para diferenciar estas áreas! 
 \end{block}
\end{frame}

\begin{frame}{}
\frametitle{Estimação de parâmetros}
\begin{block}{}
\justifying
 Um candidato a um cargo público pode desejar estimar a verdadeira proporção de eleitores 
 a seu favor ao obter as opiniões de uma amostra aleatória de 100 eleitores. A quantidade 
 de eleitores na amostra que favorece o candidato poderia ser utilizada como uma 
 estimativa da verdadeira proporção de eleitores na população. Se tivermos conhecimento 
 da distribuição amostral da variável proporção de candidatos podemos estabelecer o 
 grau de certeza da estimação. Por isso, esse problema faz parte da área de estimação.
 \end{block}
\end{frame}

\begin{frame}{}
\frametitle{Teste de hipóteses}
\begin{block}{}
\justifying
 O mesmo candidato ao cargo público pode desejar saber se a proporção de eleitores que 
 o favorece no bairro A é maior que a proporção de eleitores que o favorece no bairro B. 
 Temos aqui um problema de teste de hipóteses.
 \end{block}
\end{frame}

\begin{frame}{}
\frametitle{Estimação de parâmetros}
\begin{block}{}
\justifying
Qual a distribuição da altura dos alunos da UFV? É razoável pensar num modelo Normal, a
questão agora é identificar os parâmetros $(\mu,\sigma^{2})$ para que a distribuição 
fique completamente especificada. Como fazer isso?
\begin{itemize}
\item Podemos medir a altura de todos os alunos da UFV? Neste caso não é necessário usar 
Inferência Estatística!\pause
 \item Ou podemos escolher estrategicamente uma amostra $(X_{1},X_{2},\cdots, X_{n})$ da 
população de alunos e através dessa amostra inferir sobre os parâmetros 
$(\mu,\sigma^{2})$ da população. O resultado vai depender da qualidade da amostra e do processo de amostragem. 
\end{itemize}
Esse é um problema básico da área de Estimação.
 \end{block}
\end{frame}

\begin{frame}{}
\frametitle{Testes de hipóteses}
\begin{block}{}
Suponha agora que desejamos saber se a média da altura dos alunos do Campus UFV - Florestal é maior que a dos alunos do Campus de Rio Paranaíba $(1,7m)$?
\begin{itemize}
\item Para tomarmos uma decisão, escolhemos estrategicamente uma amostra 
 $(X_{1},X_{2},\cdots, X_{n})$ da população de alunos do campus UFV - Florestal e através dessa amostra avaliamos se $\mu>1,7m$ com alta probabilidade?
\end{itemize}
Esse é um problema básico de Testes de hipóteses.
 \end{block}
\end{frame}

\section{Testes de hipóteses}
\begin{frame}{}
\frametitle{Testes de hipóteses}
\begin{block}{}
\justifying
Em geral, o problema enfrentado pelo cientista ou pesquisador não é tanto a estimação dos parâmetros populacionais, mas a formação de um procedimento com base 
em dados que possa produzir uma conclusão sobre algum sistema cientifico. Por exemplo:
\begin{itemize}
    \item Um pesquisador da área médica pode decidir, com base em evidências experimentais, se ingerir café aumenta o risco de câncer;\pause
    \item Um engenheiro pode ter de decidir, com base em dados amostrais, se há diferença na acurácia de dois medidores;\pause
    \item Um biólogo pode querer coletar dados para decidir se o tipo sanguíneo e cor dos olhos de uma pessoa são variáveis aleatórias independentes;
\end{itemize}
\end{block}
\end{frame}

\begin{frame}{}
\frametitle{Testes de hipóteses}
\begin{block}{}
\justifying
Em cada um desses casos, são feitas \textbf{conjecturas} sobre o problema. E a tomada de decisões deve ser realizada com base nos dados experimentais. Formalmente, 
em cada caso, a conjectura pode ser colocada na forma de hipótese estatística. Procedimentos que levam a rejeição ou não de hipóteses estatísticas compreendem 
a área de \textbf{Testes de hipóteses} que vamos relembrar a partir de agora.
 \end{block}
\end{frame}

\begin{frame}{}
\frametitle{Testes de hipóteses}
\begin{block}{}
\justifying
Muitos problemas em engenharia requerem que decidamos qual das duas afirmações competitivas acerca do valor de algum parâmetro é verdadeira. As afirmações 
são chamadas de \textbf{hipóteses}, e o procedimento de tomada de decisão sobre a hipótese é chamado de \textbf{teste de hipóteses}. Esse é um dos mais úteis 
aspectos da inferência estatística, uma vez que muitos tipos de problemas de tomada de decisão, teste, ou experimentos no mundo da engenharia podem ser formulados 
como problemas de teste de hipóteses.
\end{block}
\end{frame}

\begin{frame}{}
\frametitle{Testes de hipóteses}
\begin{block}{Exemplo prático (\cite{montgomery2016}):}
\justifying
Suponha que um engenheiro esteja projetando um sistema de escape da tripulação de uma aeronave, que consiste em um assento de ejeção e um motor de foguete 
que energiza o assento. O motor de foguete contém um propelente. Para o assento de ejeção funcionar apropriadamente, o propelente deve ter uma taxa mínima de 
queima de $50$ cm/s. Se a taxa de queima for muito baixa, o assento de ejeção poderá não funcionar apropriadamente, levando a uma ejeção não segura. Taxas maiores 
de queima podem implicar instabilidade no propelente ou um assento de ejeção muito potente, levando outra vez a insegurança da injeção. Dessa maneira, a questão 
prática de engenharia que tem de ser respondida é: a taxa média de queima do propelente é igual a $50$ cm/s ou é igual a algum outro valor 
(maior ou menor)?
\end{block}
\end{frame}

\begin{frame}{}
\frametitle{Testes de hipóteses}
\begin{block}{Hipótese Estatística:}
Uma hipótese estatística é uma afirmação sobre os parâmetros de uma ou mais populações.
 \end{block}
\end{frame}

\begin{frame}{}
\frametitle{Testes de hipóteses}
\begin{block}{}
\justifying
Considere o sistema de escape da tripulação descrito no exemplo anterior. Suponha que estejamos interessados na taxa de queima do propelente sólido. Agora, a taxa de 
queima é uma variável aleatória que pode ser descrita por uma distribuição de probabilidades. Suponha que nosso interesse esteja focado na taxa média de queima 
(um parâmetro dessa distribuição). Especificamente, estamos interessados em decidir se a taxa média de queima é ou não 50 centímetros por segundo. Podemos 
expressar isso formalmente como:
\begin{align}\label{H0eH1}
\centering
H_{0}: \mu&=50 cm/s\\
\nonumber H_{1}:\mu &\neq 50 cm/s
\end{align}
 \end{block}
\end{frame}
%\begin{frame}{}
%\frametitle{Testes de hipóteses}
%\begin{block}{}
%Em algumas situações, podemos desejar formular uma hipótese alternativa unilateral, como em:
%\begin{align}
%\centering
%\nonumber  H_{0}: \mu=50 cm/s  &&                  &&H_{0}: \mu&=50 cm/s\\
                    %H_{1}:\mu> 50 cm/s  &&\textrm{ou}&&H_{1}:\mu &< 50 cm/s
%\end{align}
%\end{block}
%\end{frame}
\begin{frame}{}
\frametitle{Testes de hipóteses}
\begin{block}{}
\justifying
A afirmação $H_{0}: \mu=50$ centímetros por segundo na Equação \ref{H0eH1} é chamada de hipótese nula, e a afirmação $H_{1}:\mu \neq 50$ centímetros por segundo 
é chamada de hipótese alternativa. Uma vez que a hipótese alternativa especifica valores de $\mu$ que poderiam ser maiores ou menores do que $50$ centímetros por 
segundo, ela é chamada de hipótese alternativa bilateral.
\end{block}
\pause
\begin{block}{}
Em algumas situações, podemos desejar formular uma hipótese alternativa unilateral, como em:
\begin{flalign}
\begin{aligned} 
	\begin{cases}
H_{0}: \mu=50 cm/s\\
H_{1}:\mu> 50 cm/s
\end{cases}
\end{aligned}
\quad\textrm{ou}\quad
\begin{aligned}
\begin{cases}
H_{0}: \mu=50 cm/s\\
H_{1}:\mu< 50 cm/s
\end{cases} \\
\end{aligned}
\end{flalign}
 \end{block}
\end{frame}

\begin{frame}{}
\frametitle{Testes de hipóteses}

\begin{block}{}
\justifying
Sempre estabeleceremos a hipótese nula como uma reivindicação de igualdade. Entretanto, quando a hipótese alternativa for estabelecida com o sinal $<,$ a reivindicação 
implícita na hipótese nula será $\geq$ e quando a hipótese alternativa for estabelecida com o sinal $>,$ a reivindicação implícita na hipótese nula será $\leq.$
\end{block}
\end{frame}

\begin{frame}{}
\frametitle{Testes de hipóteses}
\begin{block}{}
\justifying
É importante lembrar que hipóteses são sempre afirmações sobre a população ou distribuição sob estudo, não afirmações sobre a amostra. O valor do parâmetro 
especificado da população na hipótese nula (50 centímetros por segundo no exemplo anterior) é geralmente determinado em uma das três maneiras. 
\begin{enumerate}
\item experiência passada, conhecimento do processo ou experimentos prévios. O objetivo nesse caso é determinar se o valor do parâmetro variou;\pause
\item alguma teoria ou modelo relativo ao processo sob estudo. Aqui, o objetivo do teste é verificar a teoria ou modelo;\pause
\item considerações externas, tais como projeto ou especificações de engenharia, ou a partir de obrigações contratuais. Nessa situação, o objetivo usual é avaliar a 
correção das especificações.
\end{enumerate}
 \end{block}
\end{frame}

\begin{frame}{}
\frametitle{Testes de hipóteses}
\begin{block}{}
\justifying
\textbf{Teste de hipóteses se apoiam no uso de informações de uma amostra aleatória proveniente da população de interesse}. É importante ressaltar que a verdade ou 
falsidade de uma hipótese particular pode nunca ser conhecida com certeza, a menos que possamos examinar a população inteira. Testar uma hipótese envolve: 
\begin{itemize}
\item considerar uma amostra aleatória; \pause
\item computar uma estatística de teste a partir de dados amostrais \pause
\item e então usar a estatística de teste para tomar uma decisão a respeito da hipótese nula.
\end{itemize}
\end{block}
\end{frame}

\begin{frame}{}
\frametitle{ Testes de Hipóteses Estatísticas}
\begin{block}{}
\justifying
A hipótese nula corresponde à taxa média de queima ser igual a 50 centímetros por segundo e a alternativa corresponde a essa taxa não ser igual a $50$ centímetros 
por segundo. Ou seja, desejamos testar
\begin{align*}
\centering
H_{0}: \mu=50 cm/s &&\textrm{contra}&& H_{1}:\mu \neq 50 cm/s
\end{align*}
Suponha que uma amostra de $n = 10$ espécimes seja testada e que a taxa média $\bar{x}$ seja observada. A média amostral é uma estimativa de $\mu.$ Um valor de 
$\bar{x}$ que caia próximo a $\mu = 50$ cm/s é uma evidência de que $\mu$ é realmente $50$ cm/s. Por outro lado, 
uma média amostral que seja consideravelmente diferente de $50$ cm/s evidencia a validade da hipótese alternativa $H_{1}.$ Assim, a média amostral 
é a estatística de teste nesse caso.
\end{block}
\end{frame}

\section{tikz}
\begin{frame}{}
\frametitle{Testes de hipóteses}
\begin{block}{}
\justifying
A média amostral pode assumir muitos valores diferentes. Suponha que se $48,5 \leq \bar{x}\leq 51,5,$ não rejeitaremos a hipótese nula $H_{0}:\mu = 50$ e se 
$\bar{x} < 48,5$ ou $\bar{x} > 51,5,$ rejeitaremos a hipótese nula em favor da hipótese alternativa $H_{1}:\mu \neq 50.$ Isso é ilustrado na Figura abaixo:
 \end{block}\pause
\begin{block}{}
\begin{figure}
\centering
\begin{tikzpicture}[xscale=1.5, yscale=1.5, <->=triangle 45]
\draw [>=stealth] (-5,0) -- (5,0);
\node [below right] at (4.7,0) {$\bar{x}$} ;
\draw [-] (-2,0) -- (-2,1.5) ;
\node [below] at (-2,0) {$48.5$} ;
\node [above] at (-3.5,0.7) {Rejeita $H_{0}$};
\node [above] at (-3.5,0.2) {$\mu\neq 50$ cm/s};
\draw [-] (0,0) -- (0,0.1) ;
\node [below] at (0,0) {$50$} ;
\node [above] at (0,0.7) {Falha em rejeitar $H_{0}$};
\node [above] at (0,0.2) {$\mu= 50$ cm/s};
\draw [-] (2,0) -- (2,1.5) ;
\node [below] at (2,0) {$51.5$} ;
\node [above] at (3.5,0.7) {Rejeita $H_{0}$};
\node [above] at (3.5,0.2) {$\mu\neq 50$ cm/s};
\end{tikzpicture}
\caption{Critérios de decisão para testar $H_{0}:\mu = 50$ cm/s versus $H_{1}: \mu \neq 50$ cm/s.} \label{fig:M1}
\end{figure}
 \end{block}
\end{frame}

\begin{frame}{}
\frametitle{Testes de hipóteses}
\begin{block}{}
\justifying
Os valores de $\bar{x}$ que forem menores do que $48,5$ e maiores do que $51,5$ constituem a \textbf{região crítica} para o teste, enquanto todos os valores que estejam 
no intervalo $48,5 \leq \bar{x}\leq 51,5$ formam uma região para a qual falharemos em rejeitar a hipótese nula. Por convenção, ela geralmente é chamada de 
\textbf{região de não rejeição}. Os limites entre as regiões críticas e a região de aceitação são chamados de valores críticos. 
 \end{block}
\end{frame}

\begin{frame}{}
\frametitle{Testes de hipóteses}
\begin{block}{}
\justifying
Em nosso exemplo, os valores críticos são $48,5$ e $51,5.$ É comum estabelecer conclusões relativas à hipótese nula $H_{0}.$ Logo, rejeitaremos $H_{0}$ em favor 
de $H_{1}$, se a estatística de teste cair na região crítica e falhamos em rejeitar $H_{0}$ por sua vez se a estatística de teste cair na região de aceitação.
 \end{block}
\end{frame}

\begin{frame}{}
\frametitle{Testes de hipóteses}
\begin{block}{}
\justifying
Esse procedimento pode levar a duas conclusões erradas. Por exemplo, a taxa média verdadeira de queima do propelente poderia ser igual a 50 centímetros por segundo. 
Entretanto, para as amostras de propelente, selecionados aleatoriamente, que são testados, poderíamos observar um valor de estatística de teste $\bar{x}$ que 
caísse na região crítica. Rejeitaríamos então a hipótese nula $H_{0}$ em favor da alternativa $H_{1},$ quando, de fato, $H_{0}$ seria realmente verdadeira. Esse tipo de 
conclusão errada é chamado de \textbf{erro tipo I}.
 \end{block}
\pause
\begin{block}{Erro Tipo I}
\begin{tikzpicture}
\node[draw,align=center, fill=gray!30] at (0,-1) {A rejeição da hipótese nula $H_{0}$ quando ela for verdadeira é definida como\\ \textbf{erro tipo I}.};
\end{tikzpicture}

\end{block}
\end{frame}

\begin{frame}{}
\frametitle{Testes de hipóteses}
\begin{block}{}
\justifying
Agora, suponha que a taxa média verdadeira de queima seja diferente de 50 centímetros por segundo, mesmo que a média amostral $\bar{x}$ caia na região de 
aceitação. Nesse caso, falharíamos em rejeitar $H_{0}$, quando ela fosse falsa. Esse tipo de conclusão errada é chamado de \textbf{erro tipo II}.
\end{block}
\pause
\begin{block}{Erro Tipo II}
\begin{tikzpicture}
\node[draw,align=center, fill=gray!30] at (0,-1) {A falha em rejeitar a hipótese nula, quando ela é falsa, é definida como\\ \textbf{erro tipo II}.};
\end{tikzpicture}
\end{block}
\end{frame}

\begin{frame}{}
\frametitle{Testes de hipóteses}
\begin{block}{}
\justifying
Assim, testando qualquer hipótese estatística, quatro situações diferentes determinam se a decisão final está correta ou errada. Pelo fato de a nossa decisão estar 
baseada em variáveis aleatórias, probabilidades podem ser associadas aos erros tipo I e tipo II. A probabilidade de cometer o erro tipo I é denotada pela letra grega 
$\alpha$.
 \end{block}

\begin{block}{}
\begin{center}
\begin{table}[]
\begin{tabular}{c|c|c}
&&\\
                          \textbf{Decisão}                   &\textbf{$H_{0}\,$ é verdadeira}                         &\textbf{$H_{0}$ é falsa}\\ \hline
 \multirow{2}{*}{\textbf{Não rejeita $H_{0}$}}&Correta                                                               &Erro Tipo II\\
                                                                        &\textbf{Probabilidade $=\left(1-\alpha \right)$} &\textbf{Probabilidade $=\beta$}\\ \hline
\multirow{2}{*}{\textbf{Rejeita $H_{0}$}}       &Erro Tipo I                                                          &  Correta\\
                                                                        &\textbf{Nível de significância $\alpha$}            &\textbf{Poder $=\left( 1-\beta\right)$}\\ \hline
\end{tabular}
\end{table}
\end{center}
\end{block}

\end{frame}

\begin{frame}{}
\frametitle{Testes de hipóteses}
\begin{block}{}
\justifying
A probabilidade do erro tipo I é chamada de \textbf{nível de significância}, ou \textbf{erro $\alpha$}, ou \textbf{tamanho do teste}. No exemplo da taxa de queima de 
propelente, um \textbf{erro tipo I} ocorrerá quando $ \bx> 51,5$ ou $\bx < 48,5,$ quando a taxa média verdadeira de queima do propelente for $\mu = 50$ cm/s. 
\end{block}
\end{frame}

\begin{frame}{}
\frametitle{Testes de hipóteses}
\begin{block}{}
\justifying
Suponha que o desvio-padrão da taxa de queima seja $\sigma = 2,5$ centímetros por segundo e que a taxa de queima tenha uma distribuição para a qual as condições do 
\textbf{teorema central do limite} se aplicam; logo, a distribuição da média amostral é aproximadamente normal, com média $\mu = 50$ e desvio-padrão 
$\dfrac{\sigma}{\sqrt{n}}=\dfrac{2.5}{\sqrt{10}}=0.79$.
\end{block}
\end{frame}

\begin{frame}{}
\frametitle{Testes de hipóteses}
\begin{block}{}
A probabilidade de cometer o \textbf{erro tipo I} (ou o nível de significância de nosso teste) é igual à soma das 
áreas sombreadas nas extremidades da distribuição normal na Figura abaixo:
\end{block}
\vspace{-0.5cm}
\begin{figure}
\centering
\begin{tikzpicture}[xscale=1.5, yscale=7, declare function={stdnorm(\x) = 1/(sqrt(2*pi))*exp(-0.5*(pow(\x,2)));}]
\fill[gray!30] (-2.5,0) -- plot [domain=-2.5:-3/2, samples=50] (\x, {stdnorm(\x)}) -- (-3/2,0) -- cycle;
\fill[gray!30] (3/2,0) -- plot [domain=3/2:5/2, samples=50] (\x, {stdnorm(\x)}) -- (5/2,0) -- cycle;
\draw [thick, domain=-2.5:2.5, samples=50] plot (\x, {stdnorm(\x)});
\draw [->] (-3,0) -- (3,0) ;
\node [below right] at (3,0) {$\bar{x}$} ;
\draw [dashed] (0,0) -- (0,{stdnorm(0)}) ;
\draw [dashed] (-3/2,0) -- (-3/2,{stdnorm(-3/2)}) ;
\draw [dashed] (3/2,0) -- (3/2,{stdnorm(3/2)}) ;
\node [below] at (0,0) {$50$};
\node [below] at (-3/2,0) {$48.5$};
\node [below] at (3/2,0) {$51.5$};
\node [above] at (-3,0.1) {\small{$\alpha/2=0.0287$}};
\node [above] at (3,0.1) {\small{$\alpha/2=0.0287$}};
%\draw[->] (-2.7,0.15) .. controls (.-2,.2) .. (-1.9, 0.03);
\draw[->] (-2.2,0.15) to [out=20,in=90] (-1.9,0.02);
\draw[->] (2.2,0.15) to [out=160,in=90] (1.9,0.02);
%\draw [->,thick] (2.7,0.15) to [out=120,in=0] (2.3,0.3)
%to [out=0,in=90] (1.9,0.03);
%\draw (0,0) .. controls (0,4) and (4,0) .. (4,4)
%\draw[->] ( 3,0.15) .. controls (. 30,.2) .. (1.9, 0.03);
%\node at (1.8,{stdnorm(2.3)}) {\small{$\alpha/2$}};
%\node at (-1.8,{stdnorm(2.3)}){\small{$\alpha/2$}};
\end{tikzpicture}
\caption{Região crítica para $\Ho: \mu = 50$ versus $\Hi: \mu \neq 50$ e $n = 10$}
\end{figure}
\vspace{-0.5cm}
\pause
\begin{tikzpicture}
\node[draw,align=center, fill=gray!30] at (0,-1) {$\alpha=P(\bar{X}<48.5\quad \textrm{quando}\quad \mu=50)+P(\bar{X}>51.5\quad \textrm{quando}\quad \mu=50)$.};
\end{tikzpicture}
%\begin{block}{}
%$$\alpha=P(\bar{X}<48.5\quad \textrm{quando}\quad \mu=50)+P(\bar{X}>51.5\quad \textrm{quando}\quad \mu=50)$$
%\end{block}
\end{frame}


\begin{frame}{}
\frametitle{}
\begin{block}{}
\justifying
Os valores de z que correspondem aos valores críticos $48,5$ e $51,5$ são

\begin{align*}
z_{1}=\dfrac{\bx-\mu}{\dfrac{\sigma}{\sqrt{n}}}=\dfrac{48.5-50}{0.79}=-1.9\quad \textrm{e}\quad z_{2}=\dfrac{\bx-\mu}{\dfrac{\sigma}{\sqrt{n}}}=\dfrac{51.5-50}{0.79}=1.9
\end{align*}
Logo,

$$\alpha=P(z<-1.90)+P(z>1.90)=0.0287+0.0287=0.0574$$

Essa é a probabilidade do erro tipo I. Isso implica que $5,74\%$ de todas as amostras aleatórias conduziriam à rejeição da hipótese $\Ho: \mu = 50$ cm/s, quando a 
taxa média verdadeira de queima fosse realmente 50 centímetros por segundo.Da inspeção da Figura anterior, notamos que podemos reduzir $\alpha$ alargando a 
região de aceitação.
\end{block}
\end{frame}

%\begin{frame}{}
%\frametitle{}
%\begin{block}{}
%\justifying
%
%\end{block}
%\end{frame}
%
%
%\begin{frame}{}
%\frametitle{}
%\begin{block}{}
%\justifying
%
%\end{block}
%\end{frame}
%
%\begin{frame}{}
%\frametitle{}
%\begin{block}{}
%\justifying
%
%\end{block}
%\end{frame}
%
%
%
%\begin{frame}{}
%\frametitle{}
%\begin{block}{}
%\justifying
%
%\end{block}
%\end{frame}
%
%\begin{frame}{}
%\frametitle{}
%\begin{block}{}
%\justifying
%
%\end{block}
%\end{frame}
%
%
%
%\begin{frame}{}
%\frametitle{}
%\begin{block}{}
%\justifying
%
%\end{block}
%\end{frame}
%
%\begin{frame}{}
%\frametitle{}
%\begin{block}{}
%\justifying
%
%\end{block}
%\end{frame}
%
%
%
%\begin{frame}{}
%\frametitle{}
%\begin{block}{}
%\justifying
%
%\end{block}
%\end{frame}
%
%\begin{frame}{}
%\frametitle{}
%\begin{block}{}
%\justifying
%
%\end{block}
%\end{frame}
%
%
%
%\begin{frame}{}
%\frametitle{}
%\begin{block}{}
%\justifying
%
%\end{block}
%\end{frame}
%
%\begin{frame}{}
%\frametitle{}
%\begin{block}{}
%\justifying
%
%\end{block}
%\end{frame}
%
%
%
%\begin{frame}{}
%\frametitle{}
%\begin{block}{}
%\justifying
%
%\end{block}
%\end{frame}
%
%\begin{frame}{}
%\frametitle{}
%\begin{block}{}
%\justifying
%
%\end{block}
%\end{frame}
%
%
%
%\begin{frame}{}
%\frametitle{}
%\begin{block}{}
%\justifying
%
%\end{block}
%\end{frame}
%
%\begin{frame}{}
%\frametitle{}
%\begin{block}{}
%\justifying
%
%\end{block}
%\end{frame}
%
%
%
%\begin{frame}{}
%\frametitle{}
%\begin{block}{}
%\justifying
%
%\end{block}
%\end{frame}
%
%\begin{frame}{}
%\frametitle{}
%\begin{block}{}
%\justifying
%
%\end{block}
%\end{frame}
%
%
%
%\begin{frame}{}
%\frametitle{}
%\begin{block}{}
%\justifying
%
%\end{block}
%\end{frame}
%
%\begin{frame}{}
%\frametitle{}
%\begin{block}{}
%\justifying
%
%\end{block}
%\end{frame}
%
%
%
%\begin{frame}{}
%\frametitle{}
%\begin{block}{}
%\justifying
%
%\end{block}
%\end{frame}
%
%\begin{frame}{}
%\frametitle{}
%\begin{block}{}
%\justifying
%
%\end{block}
%\end{frame}
%
%
%
%\begin{frame}{}
%\frametitle{}
%\begin{block}{}
%\justifying
%
%\end{block}
%\end{frame}
%
%\begin{frame}{}
%\frametitle{}
%\begin{block}{}
%\justifying
%
%\end{block}
%\end{frame}
%
%
%
%\begin{frame}{}
%\frametitle{}
%\begin{block}{}
%\justifying
%
%\end{block}
%\end{frame}
%
%\begin{frame}{}
%\frametitle{}
%\begin{block}{}
%\justifying
%
%\end{block}
%\end{frame}
%
%
%
%\begin{frame}{}
%\frametitle{}
%\begin{block}{}
%\justifying
%
%\end{block}
%\end{frame}
%
%\begin{frame}{}
%\frametitle{}
%\begin{block}{}
%\justifying
%
%\end{block}
%\end{frame}
%
%
%
%\begin{frame}{}
%\frametitle{}
%\begin{block}{}
%\justifying
%
%\end{block}
%\end{frame}
%
%\begin{frame}{}
%\frametitle{}
%\begin{block}{}
%\justifying
%
%\end{block}
%\end{frame}
%
%
%
%\begin{frame}{}
%\frametitle{}
%\begin{block}{}
%\justifying
%
%\end{block}
%\end{frame}
%
%\begin{frame}{}
%\frametitle{}
%\begin{block}{}
%\justifying
%
%\end{block}
%\end{frame}
%
%
%
%\begin{frame}{}
%\frametitle{}
%\begin{block}{}
%\justifying
%
%\end{block}
%\end{frame}
%
%\begin{frame}{}
%\frametitle{}
%\begin{block}{}
%\justifying
%
%\end{block}
%\end{frame}
%
%
%
%\begin{frame}{}
%\frametitle{}
%\begin{block}{}
%\justifying
%
%\end{block}
%\end{frame}
%
%\begin{frame}{}
%\frametitle{}
%\begin{block}{}
%\justifying
%
%\end{block}
%\end{frame}
%
%
%
%\begin{frame}{}
%\frametitle{}
%\begin{block}{}
%\justifying
%
%\end{block}
%\end{frame}
%
%\begin{frame}{}
%\frametitle{}
%\begin{block}{}
%\justifying
%
%\end{block}
%\end{frame}
%
%
%
%\begin{frame}{}
%\frametitle{}
%\begin{block}{}
%\justifying
%
%\end{block}
%\end{frame}
%
%\begin{frame}{}
%\frametitle{}
%\begin{block}{}
%\justifying
%
%\end{block}
%\end{frame}
%
%
%
%\begin{frame}{}
%\frametitle{}
%\begin{block}{}
%\justifying
%
%\end{block}
%\end{frame}
%
%\begin{frame}{}
%\frametitle{}
%\begin{block}{}
%\justifying
%
%\end{block}
%\end{frame}
%
%
%
%\begin{frame}{}
%\frametitle{}
%\begin{block}{}
%\justifying
%
%\end{block}
%\end{frame}
%
%\begin{frame}{}
%\frametitle{}
%\begin{block}{}
%\justifying
%
%\end{block}
%\end{frame}
%
%
%
%\begin{frame}{}
%\frametitle{}
%\begin{block}{}
%\justifying
%
%\end{block}
%\end{frame}
%
%\begin{frame}{}
%\frametitle{}
%\begin{block}{}
%\justifying
%
%\end{block}
%\end{frame}
%
%
%
%\begin{frame}{}
%\frametitle{}
%\begin{block}{}
%\justifying
%
%\end{block}
%\end{frame}
%
%\begin{frame}{}
%\frametitle{}
%\begin{block}{}
%\justifying
%
%\end{block}
%\end{frame}
%
%
%
%\begin{frame}{}
%\frametitle{}
%\begin{block}{}
%\justifying
%
%\end{block}
%\end{frame}
%
%\begin{frame}{}
%\frametitle{}
%\begin{block}{}
%\justifying
%
%\end{block}
%\end{frame}
%
%
%
%\begin{frame}{}
%\frametitle{}
%\begin{block}{}
%\justifying
%
%\end{block}
%\end{frame}
%
%\begin{frame}{}
%\frametitle{}
%\begin{block}{}
%\justifying
%
%\end{block}
%\end{frame}
%
%
%
%\begin{frame}{}
%\frametitle{}
%\begin{block}{}
%\justifying
%
%\end{block}
%\end{frame}
%
%\begin{frame}{}
%\frametitle{}
%\begin{block}{}
%\justifying
%
%\end{block}
%\end{frame}
%
%
%
%\begin{frame}{}
%\frametitle{}
%\begin{block}{}
%\justifying
%
%\end{block}
%\end{frame}
%
%\begin{frame}{}
%\frametitle{}
%\begin{block}{}
%\justifying
%
%\end{block}
%\end{frame}
%
%
%
%\begin{frame}{}
%\frametitle{}
%\begin{block}{}
%\justifying
%
%\end{block}
%\end{frame}
%
%\begin{frame}{}
%\frametitle{}
%\begin{block}{}
%\justifying
%
%\end{block}
%\end{frame}
%
%
%
%\begin{frame}{}
%\frametitle{}
%\begin{block}{}
%\justifying
%
%\end{block}
%\end{frame}
%
%\begin{frame}{}
%\frametitle{}
%\begin{block}{}
%\justifying
%
%\end{block}
%\end{frame}
%
%
%
%\begin{frame}{}
%\frametitle{}
%\begin{block}{}
%\justifying
%
%\end{block}
%\end{frame}
%
%\begin{frame}{}
%\frametitle{}
%\begin{block}{}
%\justifying
%
%\end{block}
%\end{frame}
%
%
%
%\begin{frame}{}
%\frametitle{}
%\begin{block}{}
%\justifying
%
%\end{block}
%\end{frame}
%
%\begin{frame}{}
%\frametitle{}
%\begin{block}{}
%\justifying
%
%\end{block}
%\end{frame}
%
%
%
%\begin{frame}{}
%\frametitle{}
%\begin{block}{}
%\justifying
%
%\end{block}
%\end{frame}
%
%\begin{frame}{}
%\frametitle{}
%\begin{block}{}
%\justifying
%
%\end{block}
%\end{frame}
%
%
%
%\begin{frame}{}
%\frametitle{}
%\begin{block}{}
%\justifying
%
%\end{block}
%\end{frame}
%
%\begin{frame}{}
%\frametitle{}
%\begin{block}{}
%\justifying
%
%\end{block}
%\end{frame}
%
%
%
%\begin{frame}{}
%\frametitle{}
%\begin{block}{}
%\justifying
%
%\end{block}
%\end{frame}
%
%\begin{frame}{}
%\frametitle{}
%\begin{block}{}
%\justifying
%
%\end{block}
%\end{frame}
%
%
%
%\begin{frame}{}
%\frametitle{}
%\begin{block}{}
%\justifying
%
%\end{block}
%\end{frame}
%
%\begin{frame}{}
%\frametitle{}
%\begin{block}{}
%\justifying
%
%\end{block}
%\end{frame}
%
%
%
%\begin{frame}{}
%\frametitle{}
%\begin{block}{}
%\justifying
%
%\end{block}
%\end{frame}
%
%\begin{frame}{}
%\frametitle{}
%\begin{block}{}
%\justifying
%
%\end{block}
%\end{frame}
%
%
%
%\begin{frame}{}
%\frametitle{}
%\begin{block}{}
%\justifying
%
%\end{block}
%\end{frame}
%
%\begin{frame}{}
%\frametitle{}
%\begin{block}{}
%\justifying
%
%\end{block}
%\end{frame}
%
%
%
%\begin{frame}{}
%\frametitle{}
%\begin{block}{}
%\justifying
%
%\end{block}
%\end{frame}
%
%\begin{frame}{}
%\frametitle{}
%\begin{block}{}
%\justifying
%
%\end{block}
%\end{frame}

\begin{frame}{}
\frametitle{Referências Bibliográficas}
\bibliography{bibliografia}
\end{frame}

\end{document}


