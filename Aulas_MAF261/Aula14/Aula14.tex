\documentclass[14pt,aspectratio=1610]{beamer}

\usepackage[brazil]{babel}
\usepackage[utf8]{inputenc}
%\UseRawInputEncoding
\usepackage[T1]{fontenc}
\usepackage{Sweave}
\usepackage{animate}
\usepackage{amsbsy}
\usepackage{amsfonts}
\usepackage{amsmath}
\usepackage{amssymb}
\usepackage{amsthm}
\usepackage[toc,page,title,titletoc]{appendix}
\usepackage[fixlanguage]{babelbib}
%\usepackage[pdftex]{color}
\usepackage{dsfont}
\usepackage{esvect}
\usepackage[labelfont=bf]{caption}
\usepackage{float}
\usepackage[Glenn]{fncychap}%Sonny %Conny %Lenny %Glenn %Renje %Bjarne %Bjornstrup
%\usepackage{geometry, calc, color, setspace}%
%\geometry{a4paper, headsep=1.0cm, footskip=1cm, lmargin=3cm, rmargin=2cm, tmargin=3cm, bmargin=2cm}
\usepackage{graphicx}
\usepackage{indentfirst}%Para indentar os parágrafos automáticamente
\usepackage{lipsum}
\usepackage{longtable}
\usepackage{mathtools}
\usepackage{listings}%Inserir codigo do R no latex
\usepackage{multirow}
\usepackage{multicol}
\usepackage{natbib}
\bibliographystyle{abbrvnat3}
\usepackage[figuresright]{rotating}
\usepackage{spalign}
%\usepackage{pgfpages}
\usepackage{pgfplots}
\usepackage{tikz}
\usepackage{color, colortbl}
\usepackage{ragged2e}%para justificar o texto dentro de algum ambiente
\definecolor{Gray}{gray}{0.9}
\definecolor{LightCyan}{rgb}{0.88,1,1}


\usepackage[all]{xy}
\usepackage{hyperref,bookmark}
\hypersetup{
  colorlinks=true,
  linkcolor=blue,
  citecolor=red,
  filecolor=blue,
  urlcolor=blue,
}

\usetheme{Boadilla}
%\usecolortheme[RGB={193,0,0}]{structure}

%\setbeamertemplate{footline}[frame number]
%\setbeamertemplate{footline}[text line]{%
%  \parbox{\linewidth}{\vspace*{-8pt}\hfill\date{}\hfill\insertshortauthor\hfill\insertpagenumber}}
\beamertemplatenavigationsymbolsempty
\renewcommand{\vec}[1]{\mbox{\boldmath$#1$}}
\newtheorem{Teorema}{Teorema}
\newtheorem{Proposicao}{Proposição}
\newtheorem{Definicao}{Definição}
\newtheorem{Corolario}{Corolário}
\newtheorem{Demonstracao}{Demonstração}
\newcommand{\bx}{\ensuremath{\bar{x}}}
\newcommand{\Ho}{\ensuremath{H_{0}}}
\newcommand{\Hi}{\ensuremath{H_{1}}}


\apptocmd{\frame}{}{\justifying}{} % Allow optional arguments after frame.

\title{MAF 261 - Estatística Experimental}
\author{Prof. Fernando de Souza Bastos}
\institute{Instituto de Ciências Exatas e Tecnológicas\texorpdfstring{\\ Universidade Federal de Viçosa}{}\texorpdfstring{\\ Campus UFV - Florestal}{}}
\date{2018}
\newcommand\mytext{Aula 8}
\newcommand\mytextt{Fernando de Souza Bastos}
\makeatletter
\setbeamertemplate{footline}
{
  \leavevmode%
  \hbox{%
  \begin{beamercolorbox}[wd=.333333\paperwidth,ht=2.25ex,dp=1ex,center]{author in head/foot}%
    \usebeamerfont{author in head/foot}\mytext
  \end{beamercolorbox}%
  \begin{beamercolorbox}[wd=.333333\paperwidth,ht=2.25ex,dp=1ex,center]{title in head/foot}%
    \usebeamerfont{title in head/foot}\mytextt
  \end{beamercolorbox}%
  \begin{beamercolorbox}[wd=.333333\paperwidth,ht=2.25ex,dp=1ex,right]{date in head/foot}%
    \usebeamerfont{date in head/foot}\insertshortdate{}\hspace*{2em}
    \insertframenumber{} / \inserttotalframenumber\hspace*{2ex} 
  \end{beamercolorbox}}%
  \vskip0pt%
}
\makeatother


\providecommand{\arcsin}{} \renewcommand{\arcsin}{\hspace{2pt}\textrm{arcsen}}
\providecommand{\sin}{} \renewcommand{\sin}{\hspace{2pt}\textrm{sen}}
%\newtheorem{Teorema}{Teorema}
%\newtheorem{Proposicao}{Proposição}
%\newtheorem{Definicao}{Definição}
%\newtheorem{Corolario}{Corolário}
%\newtheorem{Demonstracao}{Demonstração}

% Layout da pagina
\hypersetup{pdfpagelayout=SinglePage}
\begin{document}
\Sconcordance{concordance:Aula14.tex:Aula14.Rnw:%
1 318 1}


\frame{\titlepage}

\begin{frame}{}
\frametitle{\bf Sumário}
\tableofcontents
\end{frame}

\section{Delineamento em Quadrado Latino}
\begin{frame}{Delineamento em Quadrado Latino}
\frametitle{}
\begin{block}{}
\justifying
No Delineamento em Quadrado Latino (DQL), além dos princípios da repetição e
da casualização, é utilizado também duas vezes o princípio do controle na casualização para controlar o efeito de dois fatores perturbadores que causam variabilidade entre as unidades experimentais. Para controlar esta variabilidade, é necessário dividir as unidades experimentais em blocos homogêneos de unidades experimentais em relação a cada fator perturbador.
\end{block}
\end{frame}

\begin{frame}{}
\frametitle{}
\begin{block}{}
\justifying
O número de blocos para cada fator perturbador deve ser igual ao número de tratamentos. Por exemplo, se no experimento estão sendo avaliados $I$ tratamentos, deve ser formado para cada fator perturbador $I$ blocos e cada um destes blocos deve conter $I$ unidades experimentais. Ao final são necessários $I^{2}$ unidades experimentais. Cada uma destas $I^{2}$ unidades experimentais é classificada segundo cada um dos dois fatores perturbadores.
\end{block}
\end{frame}

\begin{frame}{}
\frametitle{}
\begin{block}{}
\justifying
Uma vez formados os blocos, distribui-se os tratamentos ao acaso com a restrição
que cada tratamento seja designado uma única vez em cada um dos blocos dos dois
fatores perturbadores.

\end{block}
\end{frame}

\begin{frame}{}
\frametitle{}
\begin{block}{}
\justifying
Geralmente, na configuração de um experimento instalado segundo o DQL, os níveis
de um fator perturbador são identificados por linhas em uma tabela de dupla entrada e os níveis do outro fator perturbador são identificados por colunas na tabela.
\end{block}
\end{frame}

\section{Exemplos}
\begin{frame}{Exemplos}
\frametitle{}
\begin{block}{}
\justifying
Num experimento com suínos pretende-se testar as rações $A,B,C$ e $D$ em 4 raças e 4 idades de animais. Sendo interesse fundamental o comportamento dos 4 tipos de rações, toma-se a raça e a idade como blocos, ou seja:
\begin{table}[!h]
\begin{tabular}{ccccc}
\hline
&\multicolumn{4}{c}{Raça}\\
\cline{2-5}
Idade&$R_{1}$&$R_{2}$&$R_{3}$&$R_{4}$\\
\hline
$I_{1}$&A&B&D&C\\
$I_{2}$&B&C&A&D\\
$I_{3}$&D&A&C&B\\
$I_{4}$&C&D&B&A\\
\hline
\end{tabular}
\end{table}
\end{block}
\end{frame}

\begin{frame}{Exemplos}
\frametitle{}
\begin{block}{}
\justifying
Num laboratório devem ser comparados 5 métodos de análise $(A, B, C, D, E),$ programados em 5 dias úteis e, em cada dia, é feita uma análise a cada hora, num
período de 5 horas. O quadrado latino assegura que todos os métodos sejam
processados, uma vez em cada período e em cada dia. O croqui abaixo ilustra a
configuração a ser adotada.
\vspace{-0.5cm}
\begin{table}[!h]
\begin{tabular}{cccccc}
\hline
&\multicolumn{5}{c}{Dia}\\
\cline{2-6}
Período&1&2&3&4&5\\
\hline
$1$&A&E&C&D&B\\
$2$&C&B&E&A&D\\
$3$&D&C&A&B&E\\
$4$&E&D&B&C&A\\
$5$&B&A&D&E&C\\
\hline
\end{tabular}
\end{table}
\vspace{-0.5cm}
Note que os níveis de uma fonte formam as linhas e os níveis da outra fonte
formam as colunas
\end{block}
\end{frame}

\section{Modelo Estatístico}
\begin{frame}{Modelo Estatístico}
\frametitle{}
\begin{block}{}
\justifying
Para o DQL o modelo estatístico é:
\begin{eqnarray*}
Y_{ij(k)} &=& \mu + I_i + c_j + t_k + e_{ij(k)}
\end{eqnarray*}
onde $Y_{ij(k)}$ é o valor esperado para a variável resposta obtido para o k-ésimo tratamento, na i-ésima linha e na j-ésima coluna, $\mu$ é a média de todos os valores possíveis da variável resposta, $I_i$ é o efeito da linha $i,\ c_j$ é o efeito da linha coluna $j,\ t_k$ é o efeito do tratamento $k$ e $e_{ij(k)}$ é o erro experimental.
\end{block}
\end{frame}

\begin{frame}{}
\frametitle{}
\begin{block}{}
\justifying
Considerando $L_i$ o total da linha i, $C_j$ o total da coluna j, $T_k$ o total do tramento k, $G$ o total geral e $C = \frac{G^2}{I^2}$, temos:
\begin{eqnarray*}
SQTotal  &=& \sum_{i,j=1}^{I} Y_{ij}^2 - C \\
SQLinhas &=& \frac{1}{I} \sum_{i=1}^{I} L_i^2 - C \\
\end{eqnarray*}

\end{block}
\end{frame}

\begin{frame}{}
\frametitle{}
\begin{block}{}
\justifying
\begin{eqnarray*}
SQColunas &=& \frac{1}{I} \sum_{j=1}^{I} C_j^2 - C \\
SQTrat    &=& \frac{1}{I} \sum_{j=1}^{I} T_i^2 - C \\
SQRes     &=& SQTotal - SQLinhas - SQColunas - SQTrat
\end{eqnarray*}

\end{block}
\end{frame}

\begin{frame}{}
\frametitle{}
\begin{block}{}
\justifying
Admitindo-se I tratamentos, conseqüentemente, I linhas e I colunas, o esquema da
análise de variância fica:
\begin{table}[!h]
\scalebox{0.7}{%
\setlength{\arrayrulewidth}{2pt}
\begin{tabular}{cccccc}
\hline
FV&GL&SQ&QM&F&$F_{tab;\alpha}$\\
\hline
&&&&&\\
 Linhas&$(I-1)$&SQLinhas & $-$ & $-$ & $-$\\
&&&&&\\
Colunas&$(I-1)$&SQColunas& $-$ & $-$ & $-$\\
&&&&&\\
Tratamentos&$(I-1)$&SQTrat&$\dfrac{SQTrat}{I-1}$&$\dfrac{QMTrat}{QMRes}$&$[(I-1);(I-1)(I-2)]$\\
&&&&&\\
Resíduo&$(I-1)(I-2)$&SQRes&$\dfrac{SQRes}{(I-1)(I-2)}$&$-$&$-$\\
\hline
Total&$I^{2}-1$&SQTotal&$-$&$-$&$-$\\
\hline
\end{tabular}
}
\end{table}
\end{block}
\end{frame}

\begin{frame}{}
\frametitle{}
\begin{block}{Exemplo (Exercício 7.1, pág. 68):}
\justifying
Num experimento de competição de variedades de cana forrageira foram usadas 5 variedades: A, B, C, D e E, dispostas em um 
quadrado latino de 5 por 5. O controle feito através de blocos horizontais e verticais teve por objetivo eliminar influências devidas a diferenças de fertilidade em duas direções. As produções, em kg por parcela, foram as seguintes:
\begin{table}[!h]
\begin{tabular}{ccccccc}
\hline
&\multicolumn{5}{c}{Colunas}\\
\cline{2-6}
Linhas&1&2&3&4&5&Totais\\
\hline
$1$&432(D)&518(A)&458(B)&583(C)&331(E)&2322\\
$2$&724(C)&478(E)&524(A)&550(B)&400(D)&2676\\
$3$&489(E)&384(B)&556(C)&297(D)&420(A)&2146\\
$4$&494(B)&500(D)&313(E)&486(A)&501(C)&2294\\
$5$&515(A)&660(C)&438(D)&394(E)&318(B)&2325\\
\hline
Totais&2654&2540&2289&2310&1970&11763\\
\hline
\end{tabular}
\end{table}
\end{block}
\end{frame}

\begin{frame}{}
\frametitle{}
\begin{block}{}
\justifying
Considerando $\alpha = 5\%$, pede-se:
\begin{itemize}
  \item Análise da variância
  \item Qual a variedade a ser recomendada, utilizando o teste de Tukey, se necessário
\end{itemize}
\end{block}
\end{frame}

\end{document}
